\section{Potential Application: Multi-Wallet Anonymous Payments}\label{sec:application}
One application scenario that we highlight is in the context of decentralized anonymous payment networks similar to Zcash \cite{zerocash}. In this example, we consider a variant of Zcash network suitably modified to work with MPC-friendly zkSNARKs discussed in this paper.
The users on the network create and consume coins via following transactions: (i) {\bf Mint} transaction which allows users to introduce coins to the system (after equivalent funds are
deposited to a backing pool), (ii) {\bf Spend} transaction which allows a user to consume his unspent coins and create new coins for other users and 
(iii) {\bf Redeem} transaction which allowed a user to redeem his unspent coins in exchange for equivalent funds in traditional banking systems. Our application 
is concerned with the {\bf Spend} transaction. In general an $n$-ary spend transaction consumes $n$ input coins and outputs $n$ coins with matching 
cumulative value. The output coins may be assigned to different users. To ensure unlinkability of input and output coins, coins are created
and spent using separate identifiers, known as {\em coin commitment} ($\mathsf{cm}$) and {\em serial number} ($\mathsf{sn}$) respectively. 
Roughly, $\mathsf{cm}$
and $\mathsf{sn}$ for a coin are linked via a trapdoor $\tau$: $\mathsf{cm} = f_1(\tau)$ and $\mathsf{sn}=f_2(\tau)$ for one-way functions
$f_1$ and $f_2$. Knowing one of them, it is infeasible to infer the other.  
As part of a spend transaction, 
the user supplies serial numbers $\mathsf{sn}_1,\ldots,\mathsf{sn}_n$ for the $n$ coins being consumed along with the commitments for the output coins.
Additionally, the 
user supplies a zero knowledge proof $\pi$ showing that each serial number corresponds to a valid and unspent coin and that the cumulative value of the coins is equal to the cumulative value of the output coins. 
By verifying the above zero knowledge proof, the participants on the network can ensure the validity of the transaction.

 The above setup does not allow different users to pool in their coins as inputs to a spend transaction, for example,
to jointly pay off another entity. The limitation stems from having to generate a zero knowledge proof, for which the user requires knowledge of all the
private information associated with a coin. One way to work around the limitation is for users to transfer their respective coins (via spend transactions) 
to one designated user, who then initiates a spend transaction consuming all the received coins on their behalf. 
However, the operation is no longer {\em atomic} and the designated user may not go ahead with his transaction. 

Using distributed proof generation, the set of pooling users supply the share of the witness corresponding to the coins they own,
and use a distributed proof generation protocol proposed in this paper to output a proof that proves the validity of all the coins to the 
larger network. This results in an atomic payment between upto $n$ senders and recipients.

Similarly, for auctions with joint bidding that use cryptocurrencies, 
a set of users can  leverage $\DPZK$ to produce a proof that they 
together own coins amounting to the submitted bid.

