
\subsection{\dpname: Distributed Prover Variant}
\label{subsec:DPgraphene}
\input{Main_DP_Graphene}
%-----------------------------------------------------------------------------
\subsection{Security Proofs}
%---------------------------------------------------------
As discussed in the definition of $\DPZK$ given in Section~\ref{sec:security model},depending on the corruption scenario there are 4 possible cases:
%-------
\begin{itemize}
	\item All the provers are corrupt and do not have a valid witness. Together they try to cheat so that the verifier accepts the proof. If a $\DPZK$ protocol withstands this corruption model, then we state that it has the {\em Soundness with Witness Extraction (SoWE)} property.
	%-------
	\item The verifier is corrupt and tries to learn about the provers' secrets. A protocol secure under this corruption scenario has the {\em Zero-Knowledge (ZK)} property.
	%-------
	\item Among all the provers, $t$ are corrupt and try to learn about the honest provers' secrets. A $\DPZK$ protocol is said to have \textit{Witness Confidentiality (WC)} property if it is secure against this corruption scenario.
	%-------
	\item A corrupt verifier colludes with $t$ corrupt provers and tries to learn honest provers' secrets. A $\DPZK$ protocol is said to have \textit{Witness Confidentiality with Collusion (WCwC)} property if it is secure against this corruption scenario.
	%-------
\end{itemize}
%-------
\noindent\textit{\textbf{Soundness with Witness Extraction:}} A $\DPZK$ protocol has {SoWE} property if there exists an (expected) polynomial-time extractor $\extrac$ that interacts with the prover on behalf of the verifier. If the verifier accepts the proof, $\extrac$ outputs a valid witness corresponding to the statement, with a very high probability. 
%-------
%------------------
\begin{lemma}
	For the protocol $\dpname$ given in Figure \ref{fig:dpgraphene}, there exists an (expected) polynomial-time extractor $\extrac$. If $\verifier$ accepts the proof given by a set of provers, then $\extrac$ can output a witness with overwhelming probability.
\end{lemma}
%-------
\begin{proof}
From Lemma~\ref{lem:graphene_sound}, we know that $\name$ has proof of knowledge property. Therefore by Lemma~\ref{lemma:generic_dpzk}, we know that $\dpname$ has the soundness with witness extraction property. This proves the above lemma.
\end{proof}
%----------------------------------

\noindent\textit{\textbf{Zero Knowledge:}}
If a verifier in a $\DPZK$ protocol learns nothing more that the assertion of the statement, then it has the zero-knowledge property.
In other words, there exists a simulator $\Sim_{DP}$ that generates a transcript $\tau$ without having a witness, such that $\tau$ is indistinguishable from a real execution of the protocol. Note that $\tau$ consists of the verifier's challenges and the provers' messages. 
%-------

\begin{lemma}\label{lem:simdpgraphene}
	There exists a simulator $\Sim_{DP}$ that outputs a perfectly indistinguishable extended view of the verifier in an honest execution of the protocol $\dpname$ for $t\leq b$.
\end{lemma}
%-------
\begin{proof}
From the Lemma~\ref{lem:simgraphene}, we know that $\name$ has the zero-knowledge property. Coupled with Lemma~\ref{lemma:generic_dpzk}, this ensures the zero-knowledge property of $\dpname$. Thus there exists a polynomial-time simulator that generates a transcript which is indistinguishable from the real transcript.
\end{proof}
%-------
%-------------------------

\noindent\textit{\textbf{Witness Confidentiality and Witness Confidentiality with Collusion:}} Our work is mainly concerned with building a public coin proof system, specifically NIZK so that anyone can verify the proof, and any prover can play a verifier's role yet learn nothing. Therefore witness confidentiality and witness confidentiality with collusion are the same in this setting.
%-------

Though the proof of the witness confidentiality property is similar to the witness confidentiality proof of Lemma~\ref{lemma:generic_dpzk}, we provide detailed proof for our protocol here.

We will separate the provers into two classes: honest($H$) and corrupt($C$). Note that $|H|+|C|=\Num$ and $|C|<t$, for some $t<\Num$. In linear check, since there is no interaction among the provers other than the aggregator, it suffices to generate the view of the aggregator. For the quadratic check, we need that the multiplications among the provers are executed securely.
%-------
\begin{lemma}\label{lem:WHlin}
	For the Distributed Linear Check protocol (Figure~\ref{fig:distlincheck}), there exists a polynomial-time simulator $\Sim$ corresponding to a subset of $t$ provers out of $\Num$ ($t < \Num$) controlled by a semi-honest PPT adversary $\Adv$ such that $\Sim$ generates a view of $\Adv$ which is indistinguishable from a real view of $\Adv$.
\end{lemma}
%-------
\begin{proof}
	Let $C$ be the set of all corrupt provers and $H$ be the set of all honest provers.
	%-------
	
	Case 1: Let the output of the protocol be ``reject''.$\Sim$ picks all the components of the view of corrupt parties uniformly at random from the appropriate domains such that it is indistinguishable from an honest execution.
	%-------
	
	Case 2: Let the output of the protocol be ``accept''. Then input for $\Sim$ consists of $\shr{\ewit} \; \forall \xi\in C$ and shares of $0^{2m-1}$. Without loss of generality, we can assume that there are 2 parties, $C$ and $H$. Also, assume that the adversary controls the aggregator. The view of $\Adv$ consists of:
	%-------
	
	\noindent verifier's randomness: $\{\rho, r, Q, \delta, \beta\}$, where $Q=\{(j_u,k_u):u\in[t]\}$
	%-------
	
	\noindent commitments: $\honshr{\comoracle}$, $\honshr{\tilde{c}_1},\ldots,\honshr{\tilde{c}_{\ell}},$ $\honshr{c_0}, \honshr{c_1}, \ldots,$ $\honshr{c_{s+\ell-1}}, \honshr{d_0}$
	%-------
	
	\noindent vectors: $\honshr{\ewit}[\cdot,j_u,k_u], \honshr{P}[\cdot,k_u], \honshr{z} \text{ and } \honshr{V_u}$
	%-------
	
	$\Sim$ does the following: 
	%-------
	\begin{itemize}
		%-------
		\item[--] $\Sim$ picks uniformly at random the challenges on behalf of the verifier from the respective domains i.e. $\{\rho, r, Q=\{(j_u,k_u):u\in[t]\}, \delta, \beta\}$.
		%-------
		\item[--] Then $\Sim$ picks $\ewit[\cdot,\cdot,k_u]$ such that $\ewit[i,\cdot, k_u] \in \rsc{\alpha}{h,2m-1}$ and computes $\honshr{\ewit}[\cdot, \cdot, k_u] = \ewit[\cdot,\cdot, k_u] - \corshr{\ewit}[\cdot, \cdot,k_u]$.
		%-------
		\item[--] $\Sim$ computes $\honshr{\comoracle}[\cdot,k_u]$ which is commitment of $\honshr{\ewit}[\cdot, \cdot, k_u]$
		%-------
		\item[--] $\Sim$ computes $\honshr{\tilde{\ewit}}[\cdot,k_u] = \sum_{i \in [p]} \rho_i\honshr{\ewit}[i,\cdot,k_u]$ and $\honshr{\tilde{c}_{k_u}}$ which is commitment of $\honshr{\tilde{\ewit}}[\cdot,k_u]$, for all $u\in[t]$.
		%-------
		\item[--] $\Sim$ picks $\honshr{\tilde{c}_1},\ldots,\honshr{\tilde{c}_{\ell}}$ such that\\
		$\honshr{\tilde{c}_{k_u}} = \prod_{a\in[\ell]} (\honshr{\tilde{c}_{a}})^{\Lambda_{n,\ell}^T[a,k_u]} \; \forall u\in[t]$. $\Sim$ can do this efficiently since $\Lambda_{n,\ell}^T$ is a full rank matrix.
		%-------
		\item[--] $\Sim$ picks $\honshr{\comoracle}[\cdot,k]$ such that\\
		$\honshr{\tilde{c}_k} = \prod_{i\in[p]} (\honshr{\comoracle}[i,k])^{\rho_i} \; \forall k\notin \{k_u:u\in[t]\}$. 
		%-------
		\item[--] $\Sim$ sends $\honshr{\comoracle}, \honshr{\tilde{c}_{1}}, \ldots, \honshr{\tilde{c}_{\ell}}$ to the aggregator.
		%-------
		\item[--] $\Sim$ computes\\
		$\honshr{P}[j,k_u] = \sum_{i\in[p]} R^i(\alpha_j, \eta_{k_u})\cdot \honshr{U}[i,j,k_u] \; \forall u\in[t]$ and $\honshr{c_{k_u}}$ which is commitment of $\honshr{P}[\cdot,k_u]$.
		%-------
		\item[--] $\Sim$ picks $\honshr{c_0},\;\honshr{d_0}$ uniformly at random and picks $\honshr{c_1}, \ldots, \honshr{c_{s+\ell-1}}$ subject to the following constraints:\\
		$\honshr{c_{k_u}} = \prod_{a\in[s+\ell-1]}(\honshr{c_a})^{\Lambda_{n,s+\ell-1}^T[a,k_u]} \; \forall u\in[t]\\
		\honshr{\cm} = (\honshr{c_0})^{\beta}\prod_{a\in[s+\ell-1]} (\honshr{c_a})^{\varphi_a}$\\
		$\Sim$ can efficiently perform this since $\Lambda_{n,s+\ell-1}^T$ is a full rank matrix.
		%-------
		\item[--] $\Sim$ sends $\honshr{c_0}, \honshr{c_1},\ldots, \honshr{c_{s+\ell-1}}, \honshr{d_0}$ to the aggregator.
		%-------
		\item[--] According to the challenge $Q$, $\Sim$ sends $\honshr{\ewit}[\cdot, j_u,k_u]$, $\honshr{P}[\cdot,k_u]$ to the aggregator.
		%-------
		\item[--] $\Sim$ computes $\honshr{z}$ in the following way:
		$\Sim$ computes \\
		%\begin{align*}
		$\sum_{j\in[m]}\corshr{z}[j] 
		=\sum_{j\in[m]} (\beta \corshr{P_0}+\corshr{\overline{P}}\varphi + \corshr{0})[j]$.
		%\end{align*} 
		$\Sim$ knows $\corshr{0}$ and from $\corshr{\ewit}$ and $r$, $\Sim$ can obtain $\corshr{P}$ completely. Only unknown term is $\corshr{P_0}$, but note that $\sum_{j\in[m]}\corshr{P_0}[j] = 0$, hence $\Sim$ can compute $\sum_{j\in[m]}\corshr{z}[j] = L$ (say).
		$\Sim$ picks $\honshr{z}$ uniformly from $\FF^{2m-1}$ satisfying $\sum_{j\in[m]} \honshr{z}[j]= r^Tb-L$.
		%-------
		\item[--] Corresponding to the challenge $\delta$, $\Sim$ computes $\honshr{V_u} = \sum_{i\in[p]} \delta_i\cdot \honshr{\ewit}[i,\cdot, k_u]$.
		%-------
		\item[--] Finally, $\Sim$ sends $\honshr{z}, \honshr{V_u}$ to the aggregator.
		%-------
	\end{itemize}
	%-------
	The transcript generated by $\Sim$ is perfectly indistinguishable from an honest execution of the protocol. Hence the linear check described in Figure~\ref{fig:distlincheck} has witness confidentiality property.
\end{proof}
%-------

The distributed quadratic check protocol described in Figure~\ref{fig:distquadcheck} has witness confidentiality property if the multiplication of the secrets held by the provers is performed securely. Suppose $\Pi_{\sf Mult}$ be a protocol that securely realizes $\FMult$ functionality, which can withstand $t$ corrupt parties. Distributed quadratic check has witness confidentiality property against $t$ corrupt provers. The following lemma ensures the claim. 
%-------
\begin{lemma}\label{lem:WHquad}
	Let $\Pi_{\sf Mult}$ be a $\Num$-party $t$-secure protocol where the inputs of the $\xi$th party are $\shr{\bf a}, \shr{\bf b}$ and the protocol outputs $\shr{(\sum_{\xi\in[\Num]} \shr{\bf a}) \cdot (\sum_{\xi\in[\Num]} \shr{\bf b})}$ to $\distprover$, where ${\bf a}, {\bf b}$ are vectors of length $2m-1$. Then corresponding to a subset of $t$ provers corrupted semi-honestly by a PPT adversary $\Adv$, there exists a polynomial-time simulator $\Sim$ that takes private values of the provers controlled by $\Adv$ as input and outputs a view which is indistinguishable from the view of $\Adv$ in the Distributed Quadratic Check protocol (Figure~\ref{fig:distquadcheck}).
\end{lemma}
%-------
\begin{proof}
	Since $\Pi_{\sf Mult}$ is a secure protocol, there exists a simulator $\Sim_{M}$ which can generate a view that is indistinguishable from an honest execution of the protocol. $\Sim$ takes inputs of the corrupt parties and output of the protocol to simulate the view. Similar to the Lemma~\ref{lem:WHlin}, we consider $C$ as the set of corrupt parties and $H$ as the set of honest parties. We follow the same notations.
	
	The view of $\Adv$ in the distributed quadratic check consists of :
	
	\noindent Verifier's Randomness: $\rho, r, Q, \tau, \gamma, \beta, \delta, \beta_x, \beta_y, \beta_z$
	
	\noindent Commitments: $\honshr{\comoracle_a}\; \forall a\in \{x,y,z\}$, $\honshr{\tilde{c_1}}$, $\ldots,$ $\honshr{\tilde{c}_{\ell}}$, $\honshr{c_0}$, $\ldots, \honshr{c_{2\ell-1}}$
	
	\noindent Vectors:$\honshr{\ewit_a}[\cdot,\cdot,k_u] \; \forall a\in\{x,y,z\}$, $ \honshr{P}[\cdot,k_u]$, $\honshr{z}$, $\honshr{V_u}$.
	
	$\Sim$ does the following:
	\begin{itemize}
		%-------
		\item[--] $\Sim$ picks $\rho, r, Q, \tau, \gamma, \beta, \delta, \beta_x, \beta_y, \beta_z$ uniformly at random from their respective domains.
		%-------
		\item[--] $\Sim$ picks $\ewit_a[\cdot,\cdot, k_u]\in \rsc{\alpha}{h,2m-1}$ for all $a\in\{x,y,z\}$  and computes $\honshr{\ewit_a}[\cdot,\cdot,k_u]= \ewit_a[\cdot, \cdot, k_u] - \corshr{\ewit_a}[\cdot, \cdot, k_u]$ $\forall a\in\{x,y,z\},\; u\in[t]$.
		%-------
		\item[--] $\Sim$ computes $\honshr{\comoracle_a}[\cdot,k_u]$ which is commitment of $\honshr{\ewit_a}[\cdot, \cdot,k_u]$ $\forall u\in[t], a\in\{x,y,z\}$.
		%-------
		\item[--] $\Sim$ computes $\honshr{\tilde{\ewit}}[\cdot,k_u]=\sum_{i\in[p]} \rho_i \honshr{\ewit_x}[i,\cdot,k_u]+ \rho_{p+i} \honshr{\ewit_y}[i,\cdot,k_u]+ \rho_{2p+i} \honshr{\ewit_z}[i,\cdot,k_u]$ and $\honshr{\tilde{c_{k_u}}}$, which is commitment of $\honshr{\tilde{\ewit}}[\cdot,k_u]$, for all $u\in [t]$.
		%-------
		\item[--] $\Sim$ picks $\honshr{\tilde{c_1}}, \ldots, \honshr{\tilde{c}_{\ell}}$ such that 
		%-------
		$\honshr{\tilde{c}_{k_u}} = \prod_{a\in[\ell]} (\honshr{\tilde{c}_{a}})^{\Lambda_{n,\ell}^T[a,k_u]} \; \forall u\in[t]$.
		$\Sim$ can pick such $\tilde{c}$ efficiently due to the fact that $\Lambda_{n,\ell}^T$ is a full rank matrix.
		%-------
		\item[--] $\Sim$ picks $\honshr{\comoracle_a}[\cdot, k]$ such that
		
		$\honshr{\tilde{c_k}} = \prod_{i\in[p]} (\honshr{\comoracle_x}[i,k])^{\rho_i}\cdot(\honshr{\comoracle_y}[i,k])^{\rho_{p+i}}\cdot(\honshr{\comoracle_z}[i,k])^{\rho_{2p+i}}$ $\forall k\notin\{k_u:u\in[t]\}$.
		%-------
		\item[--] $\Sim$ sends $\honshr{\comoracle_a}$ for all $a\in\{x,y,z\}$ and $\honshr{\tilde{c_1}}, \ldots, \honshr{\tilde{c}_{\ell}}$ to the aggregator.
		%-------
		\item[--] $\Sim$ picks $z$ from $\FF^{2m-1}$ such that $z[j]=0\; \forall j\in[m]$. Then $\Sim$ splits $z$ into $\honshr{z}$ and $\corshr{z}$ such that $\honshr{z}+\corshr{z} = z$.
		%-------
		\item[--] $\Sim$ computes $\corshr{\ewit_x\cdot\ewit_y}[\cdot,\cdot,k_u]$ from $\corshr{\ewit_x}, \corshr{\ewit_y}$ and $\honshr{\ewit_x}[\cdot,\cdot,k_u], \honshr{\ewit_y}[\cdot,\cdot,k_u]$ and picks $\corshr{\ewit_x\cdot\ewit_y}[i,\cdot,k]$ $\in \rsc{\alpha}{h,2m-1}$ uniformly for $k\notin \{k_u:u\in[t]\}$ and compute such that
		
		$\{(\sum_{i\in[p]}r_i\corshr{\ewit_x\cdot\ewit_y}[i,\cdot,\cdot])\varphi\}[j]=z[j] -\honshr{z}[j]+\{(\sum_{i\in[p]}r_i\corshr{\ewit_z}[i,\cdot,\cdot])\varphi\}[j]$ $\forall j\in[m]$.
		It is easy to see that picking such $\corshr{\ewit_x\cdot\ewit_y}[\cdot,\cdot,k]$ can be done efficiently.
		%-------
		\item[--] $\Sim$ calls $\Sim_{M}$ on input $\corshr{\ewit_x}, \corshr{\ewit_y}$ and $\corshr{\ewit_x\cdot\ewit_y}$ and gets a valid transcript of the interaction among the provers.
		%-------
		\item[--] $\Sim$ computes $\honshr{P}[\cdot,k_u]$ from $\corshr{\ewit_x}$, $\corshr{\ewit_y}$ and 
		
		\noindent$\honshr{\ewit_x}[\cdot,\cdot,k_u]$, $\honshr{\ewit_y}[\cdot,\cdot,k_u]$ for all $u\in[t]$ and $\honshr{c_{k_u}}$, commitments of $\honshr{P}[\cdot,k_u]$.
		
		\item[--] $\Sim$ picks $\honshr{c_0}$, $\honshr{d_0}$ uniformly and picks
		$\honshr{c_1}, \ldots, \honshr{c_{2\ell-1}}$ such that:\\
		$\honshr{c_{k_u}} = \prod_{a\in[2\ell-1]}(\honshr{c_a})^{\Lambda_{n,2\ell-1}^T[a,k_u]}$ $\forall u\in[t]$\\
		$\cm = (\honshr{c_0})^{\beta}\prod_{a\in[2\ell-1]}(\honshr{c_a})^{\varphi_a}$.
		%-------
		\item[--] $\Sim$ sends $\honshr{c_0}, \honshr{c_1}, \ldots, \honshr{c_{2\ell-1}}, \honshr{d_0}$ to the aggregator.
		%-------
		\item[--] $\Sim$ sends $\honshr{\ewit_a}[\cdot,j_u,k_u]$ for all $a\in\{x,y,z\}$ and $\honshr{P}[\cdot,k_u]$, as responses.
		%-------
		\item[--] $\Sim$ finally sends $\honshr{z}$ and $\honshr{V_u}$ corresponding to the challenges $\delta$ and $\beta_x, \beta_y, \beta_z$.
		%-------
	\end{itemize}
	The view generated by $\Sim$ is indistinguishable from a transcript of an honest execution.
\end{proof}
%-------
Since $\dpname$ is the combination of linear check and quadratic check, simulator for the complete protocol follows the same strategies as the simulators described in Lemma~\ref{lem:WHlin} and \ref{lem:WHquad}. Hence, the $\dpname$ (Figure~\ref{fig:dpgraphene}) has the witness confidentiality property.

\begin{lemma}
	Let $\Pi_{\sf Mult}$ be a secure protocol described in Lemma~\ref{lem:WHquad} which is used in $\dpname$ (step 5), then $\exists$ polynomial-time simulator $\Sim$ such that $\Sim$ can generate a view of the adversary $\Adv$ which is indistinguishable from the view of $\Adv$ in a real execution of the protocol.
\end{lemma}

\begin{proof}
	$\Sim$ starts with picking the verifier's randomness and does the same as the simulators for the linear check (Lemma~\ref{lem:WHlin}) and the quadratic check (Lemma~\ref{lem:WHquad}). Therefore the transcript generated by $\Sim$ is indistinguishable from the adversary's view.
\end{proof}
%---------------------------------------------------------