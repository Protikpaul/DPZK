In the single-prover setting, the efficiency of a zero-knowledge protocol is usually measured in terms of: 
\begin{itemize}
	\item {\em prover complexity}, denoted by $t_\prover$, which represents the time complexity of the prover algorithm,
	\item {\em proof/argument size}, denoted by $\zkcomm$ that refers to the amount of communication from the prover to the verifier,
	\item {\em verifier complexity}, denoted by $t_\verifier$, which represents the time complexity of the verifier algorithm, and
	\item {\em round complexity}, denoted as $\zkrounds$, which represents rounds of interaction between the prover and the verifier. 
\end{itemize} 
However, these parameters do not capture the core bottleneck in the setting of multiple provers.
As a first step in our work, we identify additional parameters with significant impact in a distributed proof generation. A lower value of each parameter enhances the practicality of the distributed protocol.

\noindent\textit{Proof-generation communication ($\prcomm$)}:
This parameter quantifies the amount of communication between the provers during the distributed proof generation. 
This is meant to capture additional MPC communication that provers would incur while computing the messages to the verifier. 
Using secret-sharing based MPC protocols, ~\cite{GMW87, BGW88, SPDZ}, this depends on the number of multiplications between inputs 
from different parties ({\em cross-multiplications}) that need to be performed to compute the messages to the verifier. 


\noindent\textit{Proof generation rounds ($\prrounds$)}:
This indicates the number of MPC executions between the provers during the distributed proof generation. 
This is orthogonal to $\zkrounds$ and may not have any correlation with it. 
The parameter $\prrounds$ is also of cryptographic interest. If there are more than one round of prover message generation that requires MPC, care has to be taken to ensure a secure composition of the individual MPCs to prove the complete protocol is secure.

\noindent\textit{Shared circuit complexity}:
We will now elaborate a bit more on the number of cross-multiplications in the circuit that needs to be evaluated distributedly for proof generation since
that is of practical relevance and influences prover communication ($\prcomm$)
and proof generation rounds ($\prrounds$). In applications where the initial witness is canonically \textit{partitioned} among the provers, \UPDATED{that is the witness $\wit = \wit_1||\ldots||\wit_{\Num}$,} the term \textit{shared circuit} denotes a sub-circuit consisting of wires, whose values are functions of inputs from more than one prover. 

Consider the circuit for verifying the validity of the {\bf Spend} transaction in Section \ref{sec:application}. 
Let $\mathsf{rt}$ denote the Merkle root over the list of coins on the ledger. Thus, to show that a coin 
commitment $\mathsf{cm}$ corresponds to a valid coin, one provides an authentication path $\bm{p}$ from 
$\mathsf{cm}$ to $\mathsf{rt}$. Hence, the witness $\bm{w}_i$ involves the coin commitment $\mathsf{cm}_i$, the trapdoor $\tau_i$ for the coin and authentication path $\bm{p}_i$ from $\mathsf{cm}_i$ to the root $\mathsf{rt}$. This part of the witness is completely known to the owner of the $i^{th}$ coin. Thus the witness $\bm{w}$ authenticating the coins against the serial numbers is canonically partitioned into $\bm{w}_1,\ldots,\bm{w}_n$
and requires no interaction to compute. On the other hand, secure computation is required to compute shares of witness wires 
checking the equality $u_1+\ldots+u_k == v_1+\ldots+v_n$ where $u_i$ is the value of the $i^{th}$ output coin and $v_i$ 
denotes the value of the $i^{th}$ input coin. Clearly, this ``shared circuit'' 
constitutes only a fraction of the entire witness. It would be desirable, though not obvious if the secure computation for {\em computing the proof} 
also depends on the size of {\em shared circuit} instead of the that of the entire circuit. 
Since verifying merkle authentication paths using 
standard hash functions (like SHA2) incur upwards of $100K$ gates, the practicality of distributed proof generation in this case will be
greatly improved if its parameters (like $\prcomm$ and $\prrounds$) are 
linear in the size of the shared circuit $N_s$ and not the total circuit size $N$.
%Consider the circuit for proving validity of the {\bf Bid} transaction in Section \ref{sec:application}. We assume that $\mathsf{rt}$ denotes the Merkle
% root over the list $\mathsf{CmList}$ of coins. Thus, to show that a coin commitment $\mathsf{cm}\in \mathsf{CmList}$, one provides an authentication path 
% $\bm{p}$ from $\mathsf{cm}$ to $\mathsf{rt}$. Hence, to show that coin corresponding to each serial number
%$\mathsf{sn}_i\in \bm{s}$ appears in the list $\mathsf{CmList}$, the witness $\bm{w}_i$ involves the coin commitment $\mathsf{cm}_i$, the trapdoor $\tau_i$ for the coin 
%and authentication path $\bm{p}_i$ from $\mathsf{cm}_i$ to the root $\mathsf{rt}$. This part of the witness is completely known to the owner of 
%the $i^{th}$ coin. Thus the witness $\bm{w}$ authenticating the coins against the serial numbers is canonically partitioned into $\bm{w}_1,\ldots,\bm{w}_n$
%and requires no interaction to compute. On the other hand, secure computation is required to compute shares of witness wires evaluating the comparison
%$V==v$ where $V=v_1+\cdots+v_n$ denotes the cumulative value of the coins and $v$ denotes the public bid value. Clearly, this ``shared circuit'' 
%constitutes only a fraction of the entire witness. It would be desirable, though not obvious if the secure computation for {\em computing the proof} 
%also depends on the size of {\em shared circuit} instead of the that of the entire circuit. Since verifying merkle authentication paths using 
%standard hash functions (like SHA2) incur upwards of $100K$ gates, the practicality of distributed proof generation in this case will be
%greatly improved if its parameters (like $\prcomm$ and $\prrounds$) are 
%linear in the size of the shared circuit $N_s$ and not the total circuit size $N$.



In summary, the efficiency of a $\DPZK$ in the public-verifiable and
non-interactive setting (which is our concern) will be measured via $\prcomm$,
$\prrounds$, in addition to the parameters for the single prover protocol.

\subsection{On the formal Definition of $\DPZK$}
Just like the efficiency considerations, the security definition for a $\DPZK$ protocol needs to account for additional interaction. 
In particular, the security definition needs to capture the fact that interaction among the provers to generate the proof does not 
leak knowledge about their respective witnesses to other provers. 
We come up with an MPC-style definition based on real-world ideal world paradigm \cite{Canetti00,Goldreich2001,Lindell17,CohenL14}  
that takes the above issues into account. In the ideal-world, the provers deliver their respective part of the witnesses, 
and the functionality (that is parametrized with a language) combines them and check the assertion of a statement. 
In the real protocol, the provers participate in instances of MPC for `proof-generating functions' to generate messages for the verifier. 
To keep the proof-size independent of the number of provers, one of the provers enacts in a special role called 
aggregator that prepares the message for the verifier, taking into account communication from all its fellow provers 
and communicates the same to the verifier on behalf of all the provers. We say our protocol is secure if whatever an 
adversary (corrupting various subsets of provers and verifier) can do in real execution can be done in the ideal execution. 

We formalize the security of a $\DPZK$ protocol and state the precise trust assumptions under which we realize it in Section \ref{sec:security model}.
Here we foreshadow the important aspects: (i) {\em Witness Confidentiality}, i.e, a $\DPZK$ protocol ensures that other provers do not learn private inputs of a prover during 
distributed proof generation involving {\em semi-honest} provers, 
(ii) {\em Zero Knowledge}, i.e, an honest verifier learns nothing beyond the truth of the statement 
in a $\DPZK$ protocol involving semi-honest provers and (iii) {\em Soundness With Witness Extraction}, i.e, an honest verifier rejects a false statement even if the provers act maliciously.
In practice, the restriction of honest verifier is overcome by using standard transformations to obtain non-interactive 
argument from the interactive proof. Realizing a DPZK protocol for a malicious set of provers in an interesting future work.


\subsection{Compiler for IOP-based proof systems}
Several recent constructions of efficient SNARKs~\cite{aurora, BCS16} are modeled in terms of {\em Interactive Oracle Proofs} (IOPs). In this work, we present a compiler that compiles an existing IOP, into an IOP
that admits efficient proof generation by a distributed set of provers. This requires navigating several technical challenges: (i) IOPs use the abstraction of oracles, wherein the verifier does not receive the messages from the prover entirely, and only query a small number of positions and
(ii) transforming IOPs to {\em Non-interactive Proof of Knowledge} (SNARKs) requires the use of
collision-resistant hash functions to realize the oracles. The major challenge that we address is providing the abstraction of oracle by a distributed set of provers (with shares of the message), which is indistinguishable from the one provided by a single prover (with the complete message).

We use a common aggregator $\Ag$ to interact with the verifier on behalf of all the provers. Naively, to provide oracle access to a message, $\Ag$ itself needs to have access to the complete message. But this violates our privacy constraints: IOPs ensure privacy only when a ``small'' portion of the message is accessible. We resolve this \UPDATED{conflict} by using linear sharing and homomorphic commitments. In all rounds, provers maintain a linear sharing of the IOP message. They share commitments of their shares with $\Ag$, which then has access to the commitment of complete IOP message. To the external verifier, $\Ag$ presents the commitment as the message and provides query access to it. While answering queries on the ``committed oracle'', $\Ag$ additionally opens the commitments that are queried. We present the above construction as two steps: the first step involves transforming an existing IOP into an IOP with ``homomorphic'' oracles. In the second step, a $(\Num + 1)$-$\DPZK$ protocol is obtained where $\Num$ provers run the prover algorithm of the homomorphic IOP protocol ($\Num+1$ indicates that the protocol involves $\Num$ provers and one verifier). These provers jointly compute the proof where the provers start with a linear sharing of the witness.  We elaborate more below and defer the details to the later section (Section~\ref{sec:compiler}).
%------------------------


Let $\innp{\prover}{\verifier}$ be an $r$ round IOP-based proof system. $\prover$ sends $f_i = (m_i, \pi_i)$ as the $i$th round message and $\verifier$ has oracle access to $f_i$ where $m_i$ is given to $\verifier$ in the clear and $\verifier$ makes bounded number of queries to $\pi_i$. In the compiled protocol, say $\innp{\prover'}{\verifier'}$, $\prover'$ sends $f'_i = (m_i, \pi'_i)$ at the $i$th round where $\pi'_i$ is the commitment of $\pi_i$. In this step a homomorphic commitment is used. If $\verifier'$ queries some location to the oracle, oracle sends the committed value, and $\prover'$ provides the corresponding opening. $\verifier'$ checks if the opening of $\pi'_i$ on the queried locations are correct. If all the openings are correct then $\verifier'$ runs the verification algorithm of $\verifier$. Completeness and zero-knowledge of the newly obtained protocol follow directly from the underlying protocol, and soundness depends on the soundness property of the base protocol and the binding property of the commitment scheme.

To obtain the DPZK from the compiled protocol, we start with $\Num$ copies of $\prover'$. Each of these provers computes a linear sharing of $(m_i, \pi_i)$. If the $m_i$ and $\pi_i$ are linear functions of the secret, then no interaction among the provers is required. That is, they locally compute these shares. If not, they perform secure evaluations of these values and obtain linear shares. All the provers locally obtain sharing of $(m_i, \pi'_i)$ from linear sharing of $(m_i,\pi_i)$. $\Ag$ obtains shares of $(m_i, \pi'_i)$, combines them, and sets them as the oracle. The verification algorithm is the same as in the single prover protocol.

This compiler preserves the proof size and the number of the rounds. However, the compiled protocol has an overhead of proof generation and verification time. This overhead depends on the oracle size, the number of oracles and the number of rounds. 

\subsection{Instantiations of distributed prover zero-knowledge}
We instantiate a few distributed proof generation protocols. As discussed earlier, a protocol obtained from the above compiler is more likely to perform better in a distributed proof generation if the base protocol has less number of rounds and oracles. Thus we start with the compiled version of Ligero~\cite{ligero} since it has a constant number of rounds and only one oracle. Further, we provide a protocol $\name$ by optimizing this newly obtained protocol from Ligero.

Additionally, we discuss the cost of the distributed protocol obtained from Aurora~\cite{aurora}. This gives a perspective of the efficiency of a compiled protocol where the base protocol is a multi-oracle IOP.
We study the distributed proof generation for non-IOP protocols such as Bulletproofs~\cite{bulletproofs} and Spartan~\cite{spartan}. Bulletproofs achieve the distributed proof generation naturally without requiring any additional primitives excluding secure multiplication (which is inherent in all the instantiations).   In the distributed variant of Bulletproofs, provers securely evaluate multiplication of $2$ vectors. The size of these vectors is $O(N_s)$, where $N_s$ is the size of the shared circuit. Furthermore, each prover sends $O(N)$ field elements towards the aggregator. In the distributed version of Spartan, $O(N^2)$ multiplications are required among the provers. However, this construction does not ensure privacy among the provers. This shows the non-triviality of the distributed proof generation. 
 

\subsection{\name{}: an MPC-friendly zero-knowledge protocol}
To construct $\name$, we adjoin an additional dimension in the Ligero protocol and modify the encoding, the linear check, and the quadratic check accordingly. With this optimization, we break the $\sqrt{N}$ ($N$ is the size of the circuit) barrier of Ligero, following a similar idea of the concurrent work of Ligero++~\cite{ligero++} where succinct inner product argument is used. Ligero++ uses inner product argument from Virgo~\cite{Virgo} whereas, we use Bulletproofs'~\cite{bulletproofs} inner product argument. Furthermore, for the soundness of $\name$, we provide novel results in coding theory. By adjoining an additional dimension, our protocol gets better flexibility to trade-off between proof size and verification time. Furthermore, we organize the witness so that if the shared circuit is small, then the communication among the provers is required for $N^{1-2/c}$ multiplications. In contrast, Ligero's compiled protocol would require $N^{1/2}$ multiplications. Also in distributed version of Ligero and $\name$, each prover sends $O(\sqrt{N})$ and $O(N^{1-2/c})$ field elements towards the aggregator respectively. Note that the 3-D encoding, obtained by adjoining an additional dimension, helps in reducing the proof size while keeping the verification time low. Also, if the size of the shared circuit is small, this additional dimension aids in reducing the communication among the provers as well. The details of $\name$ are given in Appendix~\ref{app:grapehene_securityproofs}.

\subsection{Related Work}\label{sec:relatedwork}
As mentioned earlier in the introduction, Pedersen \cite{Ped92} defines the
notion of distributed proof generation and proposes a generic construction using
MPC. Over the years, works have discussed distributed proof generation for a
restricted class of predicates: Desmedt et al. \cite{DDB94} proposing proofs for
graph isomorphism, various works \cite{King05, DDS, Desmedt2011} for proposing
threshold signatures, Keller et al. \cite{EfficientTZ} for a class of sigma
protocols. 
More recently \cite{bulletproofs} presented a distributed variant of {\em range proofs} which allows
several provers to compute a common proof showing their private inputs lie within an interval. However their protocol
does not satisfy our stronger privacy requirements as the size of the proof depends on the number of provers.

Trinocchio \cite{trinocchio} proposes a distributed proof generation method for
Pinocchio  \cite{pinnochio_PHGR} but only for honest majority setting. Moreover, it assumes a trusted setup. 
The work DIZK \cite{dizk} provides a distributed implementation of proof generation to reduce proving time. There is no notion of privacy among the provers in this setting.

In this work, we study how to achieve distributed proof generation from other existing works such as Ligero~\cite{ligero}, Aurora~\cite{aurora}, Bulletproofs~\cite{bulletproofs}, Spartan~\cite{spartan} and the corresponding challenges. To attain an efficient construction, we start with a Ligero-style protocol and use a similar approach to~\cite{bootle2020linear, bootle2020zero}, where the witness is viewed as a multi-dimensional matrix. However, we restrict to 3 dimensions since, among these dimensions, only the smallest one contributes to the proof size, while the remaining two aid in better verification time. In our setting, increasing the number of dimensions beyond three leads to more complicated and costlier proof-generation and verification protocols without improving any other parameters. \cite{bootle2020zero} provides linear-time prover with poly-logarithmic verification, but a significant drawback of this work is that the soundness error is $O(1)$. Furthermore,~\cite{bootle2020linear} obtain linear-time prover by using linear-time encodable codes. For this, they use a linear code provided by~\cite{druk2014linear}, whose decoding is conjectured intractable. Due to this property, ~\cite{bootle2020linear} does not satisfy the proof of knowledge property. Moreover, constructions with higher dimensions require more rounds and oracles which incur higher overhead in the distributed proof generation. Ligero++~\cite{ligero++} overcomes Ligero's $O(\sqrt{N})$ proof size bottleneck by using Virgo~\cite{Virgo} inner product argument, which uses Aurora style proof system. Thus the distributed proof generation of Ligero++ is similar to the distributed variant of Aurora. In our construction, we use inner product argument from Bulletproofs. In Table ~\ref{tab:SPZK}, we compare the different metrics of efficiency for all the existing $\DPZK$ protocols.
%

Another recent line of works such as Wolverine~\cite{wolverine}, Mac'n'Cheese~\cite{macncheese}, LPZK~\cite{LPZK}, QuickSilver~\cite{quicksilver} focuses on the scalability of the zero-knowledge proofs. These works take the ``gate-by-gate'' paradigm by relying on Vector Oblivious Linear Evaluations (VOLEs). Wolverine provides a ZK protocol with $4$ field elements per multiplication gate. Mac'n'Cheese further improves it by providing a protocol with $3$ elements per multiplication gate. While QuickSilver and LPZK both achieve a ZK protocol with $1$ element per multiplication gate, LPZK is computationally heavier. All these works have linear communication complexity. Additionally, if the circuit satisfies a certain condition, called ``weak notion of uniformity'', QuickSilver achieves sub-linear communication. The gate-by-gate paradigm employed in these works, however, is restricted to interactive protocols. That is, it does not support non-interactiveness.
Furthermore, a common aggregator approach does not work for these protocols since this would either require the verifier to communicate with all the provers or the aggregator to learn the whole witness. Overcoming the problems mentioned above and obtaining distributed variants of these works is an interesting area to explore, which we leave as future work.



\begin{table*}[htp] 
	\centering
	\resizebox{\textwidth}{!}{
		\begin{tabular}{|c|c|c|c|c|c|c|} 
			\hline
			Protocols &  $\zkcomm$ & $\zkrounds$ & $t_\prover$ & $t_\verifier$ & $\prrounds$ & $\prcomm$ \\
			\hline
			\ADDED{D-Ligero} & \ADDED{$O(\sqrt{N})$} & \ADDED{$O(\log (N))$} & \ADDED{$O(\frac{N}{\log(N)})E + O(N\log(N))M$} & \ADDED{$O(\sqrt{N})E + O(N)M$} & \ADDED{1} & \ADDED{$O(\Num\cdot \sqrt{N} + \Num^2\cdot \max(N_s,\sqrt{N}))$}\\
			\hline
			\ADDED{D-Ligero++} & \ADDED{$O(\log^2{(N)})$} & \ADDED{$O(\log (N))$} & \ADDED{$O(N\log(N)) E+ O(N\log (N))M$} & \ADDED{$O(N\log (N))E+ O(N)M$} & \ADDED{$\log (N)$} & \ADDED{$O((\Num\cdot N + \Num^2\cdot N)\log (N))$}\\
			\hline
			\ADDED{D-Aurora} & \ADDED{$O(\log^2{(N)})$} & \ADDED{$O(\log (N))$} & \ADDED{$O(N\log (N)) E+ O(N\log (N))M$} & \ADDED{$O(N\log(N))E+ O(N)M$} & \ADDED{$\log (N)$} & \ADDED{$O((\Num\cdot N + \Num^2\cdot N)\log (N))$}\\
			\hline
			D-Bulletproofs & $O(\log (N))$ & $O(\log(N))$ & $O(N)E$ & $O(N)E$ & 1 & $ O(\Num\cdot N + \Num^2\cdot N_s)$ \\
			\hline
			\dpname & $O(N^{1/c})$ & $O(\log(N))$ & $O(\frac{N}{\log(N)})E + O(N\log(N))M$ & $O(N^{1-2/c})E + O(N)M$ & 1 & $ O(\Num\cdot N^{1-2/c} + \Num^2\cdot \max(N_s,N^{1-2/c}))$ \\
			\hline
		\end{tabular}
	}
	\caption{Comparison amongst the $\DPZK$s. Here $N$ is the size of the circuit and, $c$ is a positive integer of our choice. $\Num$ is the number provers in the $\DPZK$ setting. $M$ is to indicate the amount of computation required for a single multiplication on the field elements, and $E$ is to indicate the amount of computation required for single exponentiation on the group elements, in which $\mathsf{Dlog}$ is assumed to be hard. And $N_s$ denotes the size of the shared circuit.}\label{tab:SPZK}
\end{table*}
