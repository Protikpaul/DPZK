%---------
In this section, we present the distributed proof generation from various protocols. \UPDATED{We call the distributed variant of a protocol ``X'' as ``D-X''.} We start with Ligero~\cite{ligero}, where we discuss the efficiency of the distributed prover variant, D-Ligero. We optimize D-Ligero and provide a zero-knowledge protocol, named $\name$, that achieves better proof size, reduced verification time as well as communication among the provers. We explain the high-level idea of $\name$ in this section and provide the details in Appendix~\ref{app:grapehene_securityproofs}. The distributed version of $\name$ is given in Appendix~\ref{app:dp_grapehene_securityproofs}.
We then move on to the state-of-the-art IOP-based zero-knowledge protocol Aurora~\cite{aurora}. We discuss the efficiency of the compiled version of Aurora that enables distributed proof generation. This multi-oracle IOP shows the overhead of the compilation. 
We also discuss distributed proof generation from non-IOP protocols such as Bulletproofs~\cite{bulletproofs} and Spartan~\cite{spartan}. We study the efficiency and challenges of the distributed prover variants of these protocols. 


%In Ligero~\cite{ligero}, an extended witness is viewed as a matrix of size $O(\sqrt{N}) \times O(\sqrt{N})$. To do that, first $\ell$ entries are set as the first row of the matrix, next $\ell$ entries form the second row and so on. This way a matrix is formed which has, say $m$ rows and $\ell$ columns, where $m = \ell = O(\sqrt{N})$. We are calling this as canonical matrix form. 
%Then each row of the matrix is encoded. Then in that protocol, the prover is required to communicate messages, one of which is the size of a row and $O(1)$ many columns of the encoded extended witness. This imposes a natural restriction on Ligero as they cannot skew the dimensions to reduce the proof size. Among the two dimensions, the largest dimension becomes the dominating factor that makes $\sqrt{N}$ is the best achievable communication complexity in Ligero. This bottleneck was resolved in Ligero++~\cite{ligero++}. They replace the communication of one dimension by using the inner product argument. More precisely, they rely on the inner product argument proposed in Virgo~\cite{Virgo}, which in turn relies on the Aurora proof system. Our optimization over Ligero is in a similar direction. We present the extended witness as a cube (3D matrix). One extra dimension brings more flexibility to a better trade-off between the communication complexity and the verification time. The key differences from Ligero/Ligero++ are the following.
%(i) We use bulletproofs based inner product argument on reducing the communication complexity, and also, this is mainly used to facilitate distributed prover version that others fail to achieve.
%(ii) We split the extended witness into three dimensions that aids in verification time while retaining the communication complexity proportional to the smallest dimension.
%----------

\subsection{Protocols with single oracle}
\paragraph*{D-Ligero:}
We compile the Ligero protocol, and instantiate the functional commitment scheme $\FC$ by Pedersen vector commitment scheme. Note that this $\FC$ has homomorphic property and the $\open$ algorithm outputs the committed value ${\bf m}$ as the witness. The $\verif$ algorithm can be replaced by an inner product argument where ${\bf m}$ is the witness.
%----

The proof for an R1CS instance in Ligero constitutes of the check that verifies if the witness $\wit$ satisfies the following condition: $x = A \wit \land y = B \wit \land z = C \wit \land x \circ y = z$, where is $A, B, C$ matrices are dependent on the R1CS instance. The above check is segmented into three checks: (i) Interleaved check, (ii) Linear check, and (iii) Quadratic check. 
%---

In Ligero, $\prover$ rewrites the vectors $\wit, x, y, z$ as matrices in a canonical way, that is, the first $\ell$ entries are set as the first row of the matrix, next $\ell$ entries form the second row and so on. This way a matrix is formed which has, say $m$ rows and $\ell$ columns, where $m = \ell = O(\sqrt{N})$. This is referred to as the canonical matrix form. The prover encodes these matrices using Reed–Solomon(RS)-encoding to encode each row of the matrices.
In the compiled Ligero protocol, $\prover$ computes commitment of each column of these matrices. The commitment scheme takes $m$ length messages. In the original Ligero protocol, $t$ (depends on the query complexity) many columns and a linear combination of the rows are opened to the verifier. This enforces a bound on the proof size in Ligero. We circumvent this bound by replacing the openings with inner product checks. 
%---------------

In the interleaved check, $\prover$ proves that the rows of the oracle matrices are codewords. $\prover$ proves the claim probabilistically by showing that a random linear combination of the rows is a code word. $\verifier$ uniformly picks the coefficients for the linear combination.
Therefore, if one or more rows in the matrices are not correct codewords, then the linear combination is not a codeword with high probability. Furthermore, $\verifier$ queries the oracle with few indices (the number of queries is bounded by the security parameter $t$ for zero-knowledge) and receives the corresponding columns. Upon receiving the columns, $\verifier$ performs the same linear combination and checks that it is consistent with the row received from $\prover$.
%---
In the linear check, $\prover$ proves the knowledge of a vector $x$ such that $Ax = b$ holds for public matrix $A$ and vector $b$. This check is reduced to a probabilistic check $r^TAx = r^T b$, where $r$ is a random vector picked by $\verifier$. $\prover$ computes a polynomial $p(x)$ such that the sum of the evaluations on publicly known points is equal to $r^Tb$, and sends this $p(x)$ to $\verifier$. $\verifier$ checks if the above condition holds, and it further checks if $p(x)$ is constructed correctly from $r, A$, and $\wit$. $\verifier$ validates this by querying some locations of the oracle.
%---
In the quadratic check, $\prover$ proves that $x \circ y = z$ ($\circ$ denotes the Hadamard product of two vectors), where the corresponding encoded values are set as oracles. $\verifier$ gives a randomly sampled challenge vector $r$ to $\prover$. $\prover$ constructs a polynomial $p(x)$ using $r$ and the encoding polynomials of $x, y, z$. This polynomial should evaluate to $0$ on publicly known points. $\prover$ sends the polynomial $p(x)$ to $\verifier$. Upon receiving $p(x)$, $\verifier$ checks if the above condition is true or not. Furthermore, it checks whether $p(x)$ is correctly constructed. For this, it queries some locations of the oracles.
%---

In the compiled protocol, $\verifier$ obtains commitments corresponding to the queried columns, and instead of sending the openings of those columns, we perform the inner product argument. Since the opening of the columns is not sent, we can elongate the size of the columns without increasing the proof size. The inner product argument requires communication of $O(\log N)$ elements, where $N$ is the size of the witness vector. Therefore the newly obtained protocol has proof size $O(\ell + \log (m))$, where $\ell$ is the size of the row, and $m$ is the size of the column. However, verifying these inner products requires $O(N)$ exponentiations for a witness vector of dimension $N$. Thus increasing the column size unreasonably may result in a better proof size but increases the verification time, which is undesirable.
Furthermore, for the quadratic check protocol, the polynomial $p(x)$ depends non-linearly on the witness since it requires multiplication of the polynomials used for encoding $x, y$. Therefore, provers in $\Partyset$ engage in a secure computation protocol where they obtain an additive sharing of $p(x)$. In general, this incurs communication for computing a circuit of depth $1$ with $O(N)$ multiplications. Nevertheless, if the size of the shared circuit is sufficiently low (smaller than the row size in the canonical matrix), then it is possible to embed the shared circuit in a row. In this case, MPC is required only for the row that contains the shared circuit.

%------------------------------------------------------------
%
%Similarly, for Ligero~\cite{ligero}, Ligero++~\cite{ligero++}, our compiler provides protocols with  prover time $O(N\log N)$ field multiplication plus $O(N)$ group exponentiations, and verification time $O(N)$ field multiplication plus $O(N)$ group exponentiations. Note that the proof size of the protocols remains the same in the new compiled protocols. We call the distributed versions as D-Ligero and D-Ligero++ respectively. Since Ligero++ uses inner product of Virgo, which is similar to Aurora, therefore D-Ligero++ is similar to D-Aurora.
%%---
%
%From the above compiler, we see how we can obtain a secure DPZK protocol from Ligero. In the compiled protocol, we observe that each entry of the oracle is committed. Therefore the oracle size remains the same as the oracle size of the base Ligero protocol. However, this can be optimized further, and this optimization aids in reducing the proof size by overcoming the bottleneck of $O(\sqrt{N})$.
%%--- 
%
%The proof for an R1CS instance in Ligero constitutes of the check if the witness $\wit$ satisfies the following condition: $x = A \wit \land y = B \wit \land z = C \wit \land x \circ y = z$, where is $A, B, C$ matrices are dependent on the R1CS instance. The above check is segmented into three checks: (i) Interleaved check, (ii) Linear check, and (iii) Quadratic check. 
%%---
%
%In Ligero, $\prover$ rewrites the vectors $\wit, x, y, z$ as matrices in a canonical way, further encodes these matrices using RS-encoding by encoding each row of the matrices. Finally, it sets each column of these matrices as oracles. The oracle in Ligero is a matrix, and each location of the oracle contains a column of the matrix.
%%---
%In the interleaved check, $\prover$ proves that the rows of the oracle matrices are codewords. To do that, it provides a random linear combination of the rows of the matrices, where the coefficients for the linear combination are picked by $\verifier$. Therefore, if one or more rows in the matrices are not correct codewords, then the linear combination is not a codeword with high probability. Furthermore, $\verifier$ queries the oracle with few indices (the number of queries is bounded by the security parameter $t$ for zero-knowledge) and receives the corresponding columns. Upon receiving the columns, $\verifier$ performs the same linear combination and checks with the row received from $\prover$.
%%---
%In the linear check, $\prover$ proves the knowledge of a vector $x$ such that $Ax = b$ holds for public matrix $A$ and vector $b$. This check is reduced to a probabilistic check $r^TAx = r^T b$, where $r$ is a random vector picked by $\verifier$. $\prover$ computes a polynomial $p(x)$ such that the sum of the evaluations on publicly known points is equal to $r^Tb$, and sends this $p(x)$ to $\verifier$. $\verifier$ checks if the above condition holds or not, it further checks if $p(x)$ is constructed correctly from $r, A$, and $\wit$. To check this, $\verifier$ queries some locations of the oracle.
%%---
%In the quadratic check, $\prover$ proves that $x \circ y = z$, where the corresponding encoded values are set as oracles. $\verifier$ gives a randomly sampled challenge vector $r$ to $\prover$. $\prover$ constructs a polynomial $p(x)$ using the challenge $r$ and the polynomials used in the encoding of $x, y, z$. This polynomial should evaluate to $0$ on publicly known points. $\prover$ sends the polynomial $p(x)$ to $\verifier$. Upon receiving $p(x)$, $\verifier$ checks if the above condition is true or not. Furthermore, it checks if $p(x)$ is correctly constructed or not. For this, it queries some locations of the oracles.
%%---
%
%In D-Ligero, each entry of each oracle is committed using Pedersen commitment. $\verifier$ receives the committed values of all the entries corresponding to the queried locations in each column.
%%---
%Observe that, in this construction, we can replace the Pedersen commitment with the Pedersen vector commitment. Instead of committing to individual entries separately, we can commit to each column separately. Now, corresponding to a column, there is only one commitment, which gives an oracle, a row vector. With this modification, the verification algorithm does not change. Therefore soundness and zero-knowledge remain unaffected.
%%---  
%We further leverage the commitment scheme by using inner product arguments instead of opening columns. With this modification, we overcome the bottleneck of $O(\sqrt{N})$. This idea is in line with Ligero++. We use the inner product argument from Bulletproofs.
%%---
%
%For Interleaved check, $\verifier$ receives commitments of the chosen columns by the oracle. The linear combination check on the columns can be viewed as $\innp{r}{c}$, where $r$ is the challenge vector chosen by $\verifier$, and $c$ is the column. $\verifier$ knows the vector $r$ and the commitment of $c$, $\prover$ knows both $r$ and $c$. $\prover$ and $\verifier$ deterministically computes a commitment of $r$, then $\prover$ uses $r, c$ as a witness for the corresponding inner product argument. Note that the above setting is the same as bulletproofs inner product argument.
%%---
%For Linear check, $\verifier$ receives the polynomial $p(x)$, where $p(x) = \sum_i s_i(x) f_i(x)$, where $f_i(x)$ is the polynomial used in encoding $i$th row of $x$ and $s_i(\zeta_j) = R_{ij}$, where $R$ is the canonical matrix form of the vector $R = r^TA$ and $\zeta$'s are publicly known points used for encoding (the interpolation domain). $\verifier$ evaluates $p(x)$ on $\zeta$'s and checks if it sums up to $r^Tb$. Furthermore, it needs to check if $p(x)$ is well-formed or not, i.e. $p(\eta_j) = \sum_i s_i(\eta_j) \ewit[i,j]$, where $\eta$'s are the public points used in encoding (evaluation point), and $\ewit[\cdot,j]$ is the $j$th column in the oracle matrix. This check can be viewed as $p(\eta_j) = \innp{s_j}{\ewit[\cdot,j]}$, where $s_j = (s_1(\eta_j), \ldots, s_m(\eta_j))$, considering $m$ as the number of rows. Similar to the interleaved check, we check this using bulletproofs inner product argument.
%%---
%For Quadratic check, consider, $f^x_i(x), f^y_i(x), f^z_i(x)$ are the polynomials used in encoding of $x, y, z$ and $\ewit^x[\cdot,j], \ewit^y[\cdot,j], \ewit^z[\cdot, j]$ are the $j$ the columns of the oracles corresponding to $x, y, z$. Similar to the linear check, $\zeta$'s, $\eta$'s are the set of points used for interpolation and evaluation in the encoding respectively.  $\verifier$ gives a random vector $r$ as a challenge to $\prover$. $\prover$ computes $p(x) = \sum_i r_i [ f^x_i (x) \cdot f^y_i(x) - f^z_i(x)]$ and sends $p(x)$ to $\verifier$. Note that $p(\zeta_j) = 0$, $\verifier$ checks this. To ensure the correct computation of $p(x)$, $\verifier$ checks $p(\eta_j) = \sum_i r_i [ \ewit^x[i,j] \cdot \ewit^y[\cdot,j] - \ewit^z[\cdot,j]]$. This can be viewed as $p(\eta_j) = \innp{r}{(\ewit^x[\cdot,j] \cdot \ewit^y[\cdot,j] - \ewit^z[\cdot,j])}$. Unfortunately, we cannot use bulletproofs inner product argument directly here, since the commitment of $\ewit^x[\cdot,j] \cdot \ewit^y[\cdot,j]$ is not available to $\verifier$. To use inner product argument, we see the above equation as follows:
%%----
%\begin{align*}
%p(x) & = \innp{r}{(\ewit^x[\cdot,j] \cdot \ewit^y[\cdot,j] - \ewit^z[\cdot,j])} \\
%& = \innp{r}{(\ewit^x[\cdot,j] \cdot \ewit^y[\cdot,j])} - \innp{r}{\ewit^z[\cdot,j]} \\
%& = \innp{(r \circ \ewit^x[\cdot,j])}{ \ewit^y[\cdot,j]} - \innp{r}{\ewit^z[\cdot,j]} \\
%& = \innp{(r \circ \ewit^x[\cdot,j] || r)}{(  \ewit^y[\cdot,j] || - \ewit^z[\cdot,j])} \\
%\end{align*}
%%----
%Since, $r$ is available to both $\prover$ and $\verifier$ and the commitments of $ \ewit^x[\cdot,j],  \ewit^y[\cdot,j],  \ewit^z[\cdot,j]$ are available to both $\prover$ and $\verifier$. They obtain the commitments of both $(r \circ \ewit^x[\cdot,j] || r)$ and $ (\ewit^y[\cdot,j] || - \ewit^z[\cdot,j] )$ and perform the inner product check.
%%----
%Let $\epsilon_L$ and $\epsilon_B$ be the soundness error of Ligero and Bulletproofs inner product argument, respectively. Then, the modified protocol's soundness error is $O(\epsilon_L + \epsilon_B)$. However, zero-knowledge remains unharmed since, in the modified protocol, the verifier is not learning anything additional. 
%%----
%Note that in the modified protocol, we can elongate the column size and shorten the row size. Let the row size be $n$, and the column size be $m$, then the proof size is $O(n + \log (m))$ since the proof size of the inner product argument in bulletproofs is $\log (N)$ for $N$ sized vector.
%%---
%However, setting $m$ to be arbitrarily large increases verification time. 
%%---
%To obtain better proof size and verification time simultaneously, we optimize further by adjoining an additional dimension. The details are given in Section~\ref{sec:graphenec}.

%----
%We provide construction of a new proof system which can be obtained from Ligero/Ligero++~\cite{ligero,ligero++} using our compiler and some additional optimization. The rationale behind opting for Ligero style proof system is that it does not have many oracles. Therefore the conversion is less costly.
%%----
%Using the compiler directly on Ligero++~\cite{ligero++} gives 2D version of our construction where the Virgo~\cite{Virgo} inner product is replaced by Bulletproofs~\cite{bulletproofs} inner product. We optimize further in our construction by adjoining additional dimension which aids in better trade-off between proof size and verification time. 
%----
\CHANGED{\paragraph*{$\name$ and $\dpname$:}}
%----
%In our setting, we explored the scenarios by increasing the number of dimensions. Those approaches lead to more complicated and costlier proof-generation and verification protocols without any more improvements. \cite{bootle2020zero} provides linear-time prover with poly-logarithmic verification, but a major drawback of this work is that the soundness error is $O(1)$. Furthermore,~\cite{bootle2020linear} obtain linear-time prover by using linear-time encodable codes. For that, they use a linear code provided by~\cite{druk2014linear}, whose decoding is conjectured intractable. Due to this property, ~\cite{bootle2020linear} does not satisfy the proof of knowledge property.
%----

\CHANGED{In \name, we follow the same approach as in Ligero. We start with the extended witness $\wit$, consider it as a 3 dimensional matrix of size $p \times m \times s$ where $p$ 2 dimensional matrices of size $m \times s$, called a slice, are stacked one after another. In contrast to Ligero, where each row is encoded separately, we encode each two-dimensional slice separately. Let $\ell = s + \bi$, $h > 2m$, $n > 2\ell$ where $\bi$ is the bounded independence parameter depending on the query complexity. Let ${\bm \zeta} = \{ \zeta_1, \ldots, \zeta_{\ell}\}$, ${\bm \eta} = \{\eta_1, \ldots, \eta_{n}\}$ and ${\bm \alpha} = \{ \alpha_1, \ldots, \alpha_h \}$ be public set of points in $\FF$. We define $G = \{(\alpha_j,\zeta_k) : j \in [m],k \in [\ell] \}$ and $H = \{(\alpha_j,\eta_k) : j \in [h],k \in [n] \}$ to be the interpolation domain and evaluation domain respectively. To encode the $i$th slice, we construct a bivariate polynomial $Q^i(x,y)$ such that $Q^i(\alpha_j,\zeta_k) = \wit[i,j,k]$ and $\deg_x(Q^i) < m$ and $\deg_y(Q^i) < \ell$.
The encoded witness $\ewit$ is such that $\ewit[i,j,k] = Q^i(\alpha_j,\eta_k)$ for $i\in[p], j\in[h], k\in[n]$. Then each column of each slice is committed using Pedersen vector commitment. Here we instantiate the homomorphic $\FC$ with the Pedersen commitment scheme with message length $h$. Finally, $\verifier$ is provided oracle access to the committed matrix.
We design interleaved check protocol over committed values which is performed together with the linear check and quadratic check. The details are presented in Appendix~\ref{subsec:graphenec}. 
The linear check is similar to Ligero, where a probabilistic reduction is performed. $\verifier$ sends a random $r$ followed by both $\prover$ and $\verifier$ locally computing $R = r^TA$. Using ${\bm \alpha, \bm \eta}$ $\prover$, $\verifier$ interpolate $p$ polynomials $R^i(x,y)$ such that $R^i(\alpha_j, \zeta_k) = R[i,j,k]$ for $i\in[p], j\in[m], k\in[s]$ and $\deg_x(R^i) < m$, $\deg_y(R^i) < s$ for all $i$. $\prover$ computes polynomial $p_j(y) = \sum_{i\in[p]} R^i(\alpha_j, y) Q^i(\alpha_j, y)$. If the witness is correct then $r^TA\wit = r^Tb \Rightarrow \innp{R}{\wit} = r^T b \Rightarrow \sum_{j\in [m], k\in [s]} p_j (\zeta_k) = r^Tb$. Furthermore, $p_j(\eta_k) = \sum_{i\in[p]} R^i(\alpha_j, \eta) \ewit[i,j,k]$ due to the encoding. $\prover$ constructs a matrix $P$ of size $h \times n$ such that $P[j,k] = p_j(\eta)$ and commits to $P$. By providing partial opening using inner product arguments from Bulletproofs, $\prover$ proves that the polynomials $p_j(y)$ satisfies the following conditions:
\begin{itemize}
	%----
	\item $p_j(\eta_k) = \sum_{i\in[p]} R^i(\alpha_j, \eta) \ewit[i,j,k]$ for all $j \in [h], k \in [n]$
	%----
	\item $\sum_{j\in [m], k\in [s]} p_j (\zeta_k) = r^Tb$
	%----
\end{itemize}
%
Similar to the linear check, in the quadratic check, $\prover$ encodes $x,y,z$ to obtain $\ewit_x, \ewit_y, \ewit_z$ and lets $Q^i_x, Q^i_y, Q^i_z$  be the respective polynomials. $\prover$ constructs polynomials $Q^i$ such that $Q^i = Q^i_x\cdot Q^i_y - Q^i_z$. Since $x\circ y = z$, we have $Q^i(\alpha_j,\zeta_k) = 0$ for all $i\in[p], j\in[m], k\in[s]$. To check this, $\verifier$ sends a random vector $r\in \FF^p$ as a challenge. $\prover$ locally computes $p_j(\cdot) = \sum_{i \in [p]} r_i Q^i(\alpha_j, \cdot)$. Analogous to the linear check, $\prover$ forms $P$ matrix and commits to it, and further proves that $P$ satisfies the following:
\begin{itemize}
	%----
	\item $p_j(\eta_k) = \sum_{i\in[p]} r^i \left[\ewit_x[i,j,k] \cdot \ewit_y[i,j,k] - \ewit_z[i,j,k]\right]$ for all $j \in [h], k \in [n]$
	%----
	\item $p_j (\zeta_k) = 0$ for all $j \in [m], k \in [s]$
	%----
\end{itemize}
%----
%
To obtain $\dpname$, parties in $\Partyset$ execute secure multiplication to get an additive sharing of $\ewit_x[i,j,k] \cdot \ewit_y[i,j,k]$.The remaining steps in both linear and quadratic check do not require any interaction among the provers. Therefore, the provers interact to evaluate a circuit with $O(N)$ many multiplications with depth $1$. However, with smaller shared circuit, where the shared circuit can be embedded into a
column of a slice such that multiplication is required only for that column.
The details of the protocol are given in Appendix~\ref{app:grapehene_securityproofs}.}


\subsection{Multiple oracle IOP}
\paragraph*{D-Aurora:}
In Aurora~\cite{aurora}, the size of the oracle is $O(N)$, the proof size is $O(\log^2 N)$, and number of rounds is $O(\log N)$. Aurora is an IOP-based proof system where almost all the messages from $\prover$ to $\verifier$ are set as oracles, and $\verifier$ makes oracle-queries to complete the verification.
%----
We can convert Aurora using our compiler, where we instantiate $\FC$ using Pedersen commitment, such that it supports DPZK. The compiled version retains the oracle size, proof size, and the number of rounds. However, the prover time and the verification time increase by $O(N\log N)$ group exponentiations due to oracle construction and validation costs. The distributed version needs secure evaluation of circuits with at least depth one and $O(N)$ multiplication gates in each round.
%----
\subsection{Non-IOP protocols}
\paragraph*{D-Bulletproofs:}
In Bulletproofs~\cite{bulletproofs}, the proof of an R1CS instance is reduced to an inner product argument which is proved recursively for the succinctness of the proof size. 
%----
The prover constructs two vector polynomials $l(X), r(X)$ of degree $1$ using the witness, statement and challenges from the verifier. It commits to the quadratic polynomial $t(X) = \innp{l(X)}{r(X)}$ where $t(X) = \sum_{i=0}^{d} \sum_{j=0}^{i} \innp{l_i}{r_j} X^{i+j}$ such that $l_i$ and $r_j$ are $l(X)$'s $i$th and $r(X)$'s $j$th vector coefficients respectively.
%----
The verifier sends a random point $x$ and prover obtains ${\bm l} = l(x)$, ${\bm r} = r(x)$ and $t = t(x)$. Using the inner product argument the prover proves that $\innp{{\bm l}}{{\bm r}} = t$.

In D-Bulletproofs, provers start with additive sharings of $l(X), r(X)$ and perform secure multiplications to obtain additive sharing of $t(X)$. This requires $O(N)$ multiplications. Upon receiving $x$, each prover locally obtains a sharing of ${\bm l,\bm r}, t$ and sends these values to the aggregator. The aggregator performs the inner product argument with the verifier to complete the proof.
%
%In D-Bulletproofs, provers start with an additive sharing of the witness and locally obtain additive sharing of $l(X), r(X)$. Parties in $\Partyset$ perform secure multiplications to obtain additive sharing of $t(X)$. This requires $O(N)$ multiplications. Upon receiving $x$ from the verifier, each prover locally obtains a sharing of ${\bm l,\bm r}, t$ and sends these values to the aggregator. The aggregator performs the inner product argument with the verifier to complete the proof.
\vspace*{-.3cm}
\paragraph*{D-Spartan:}
In Spartan~\cite{spartan}, an R1CS circuit is represented by 3 matrices $A, B, C$. A witness $\wit$ for the true instance satisfies $Az  \circ Bz = Cz$, where $z = (\stmt, 1, \wit)$ for public input $\stmt$. In Spartan, prover obtains low-degree polynomial extensions of $A, B, C, z$. Upon receiving the challenges from the verifier, the prover constructs a polynomial $f$. %$\calG_{io,\tau}$. 
Then, a sum-check is performed to ensure that $f$ satisfies a consistency condition.
%on $\calG_{io,\tau}$, to verify if $\sum_{x\in\{0,1\}^s} \calG_{io,\tau} = 0$, where $s = O(\log (N))$. 

For the distributed proof generation, provers initiate the protocol with an additive sharing of the witness $\wit$ and achieve an additive sharing of $f$ by performing $O(N^2)$ secure multiplications. In Spartan, zero-knowledge is achieved by producing proofs of dot products. In the distributed setting, this either requires each prover to open its witness of the dot product to the aggregator or all the provers produce a distributed proof for the dot products. The latter case makes it a circular problem, whereas the former one has a privacy issue.
\UPDATED{Even if a similar protocol could overcome the privacy issue, it would likely still suffer from the high communication complexity of $O(N^2)$ secure multiplication.}
%For the distributed proof generation, provers initiate the protocol with an additive sharing of the witness $\wit$ and to achieve an additive sharing of $\calG_{io,\tau}$ parties in $\Partyset$ perform MPC, which requires $O(N^2)$ multiplications. In Spartan, zero-knowledge is achieved by producing proofs of dot products. In the distributed setting, this either requires each prover to open its witness of the dot product to the aggregator or all the provers produce a distributed proof for the dot products. The latter case makes it a circular problem, whereas the former one has a privacy issue. 
The above discussion further emphasizes the non-triviality of obtaining DPZK from any ZK protocol. 
\vspace*{-.3cm}
\begin{figure}[h!]
%\begin{table*}
	\centering
	\resizebox{.3\textwidth}{!}{
		%\begin{threeparttable}
		\begin{tabular}{|c|c|c|} 
			\hline
			Protocols & $\prrounds$ & $\prcomm$ \\
			\hline
			D-Ligero & 1 			& $O(\sqrt{N})$\\
			\hline
			D-Aurora & $\log (N)$ 	& $O((N\log (N))$\\
			\hline
			D-Bulletproofs  & 1 		& $ O(N_s)$ \\
			\hline
			%D-Spartan$^\ast$ & 1 		& $ O( N^2 )$ \\
			%\hline
			\dpname 		& 1 		& $ O( N^{1-2/c})$ \\
			\hline
			%&computation & communi- cation & MPC rounds& & & \\ 
			\hline
			%\hline
			%Bulletproofs & $O(nE+q(n+m)M)$ & $O(2N-1)$ & $O(\log n)$& $O(nE+$ $q(n+m)M)$ & $O(\log n)$ & $O(\log n)$  \\
			%\hline
			%Spartan & & & & & &\\
			%\hline
			%Aurora & & & & & &\\
			%\hline
			%\name2D & $O(n \log n)M$  $+nE)$ & $O(2N-1)$ & 1 &$O((nM +$ $ n^{1-1/c}E))$ & $O(n^{1/c})$ & $O(\log n)$ \\
			%\hline
			%\name3D & $O(nM+nE)$ & $O(2N-1)$& 1 & $O(nM+$ $n^{1-2/c}E)$ & $O(n^{1/c})$ & $O(\log n) $\\
			%\hline
		\end{tabular}
		%\end{threeparttable}
	}
	\caption{\footnotesize Comparison amongst the $\DPZK$s. Here $N$ is the size of the circuit and, $c$ is a positive integer of our choice. $\ast$ indicates non-private variant of Spartan.}\label{tab:DPZKwSmallN}
	%\pnote{Con}
%\end{table*}
\end{figure}
\vspace{-.5cm}

Here we provide a comparison of $\DPZK$ protocols in the setting where the size of the shared circuit is low ($N_s < \sqrt{N}$) and the number of provers is small ($O(1)$). 