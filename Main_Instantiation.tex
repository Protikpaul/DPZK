%---------
In this section, we present the distributed proof generation from various protocols. We call the distributed variant of a protocol ``X'' as ``D-X''. We start with Ligero~\cite{ligero}, where we discuss the efficiency of the distributed prover variant, D-Ligero. We optimize D-Ligero and provide a zero-knowledge protocol, named $\name$, that achieves better proof size, reduced verification time as well as communication among the provers. We explain the high-level idea of $\name$ in this section and provide the details in Appendix~\ref{app:grapehene_securityproofs}. The distributed version of $\name$ is given in Appendix~\ref{app:dp_grapehene_securityproofs}.
We then move on to the state-of-the-art IOP-based zero-knowledge protocol Aurora~\cite{aurora}. We discuss the efficiency of the compiled version of Aurora that enables distributed proof generation. This multi-oracle IOP shows the overhead of the compilation. 
We also discuss distributed proof generation from non-IOP protocols such as Bulletproofs~\cite{bulletproofs} and Spartan~\cite{spartan}. We study the efficiency and challenges of the distributed prover variants of these protocols. 

%----------

\subsection{Protocols with single oracle}
\paragraph*{D-Ligero:}
We compile the Ligero protocol, and instantiate the functional commitment scheme $\FC$ by Pedersen vector commitment scheme. Note that this $\FC$ has homomorphic property and the $\open$ algorithm outputs the committed value ${\bf m}$ as the witness. The $\verif$ algorithm can be replaced by an inner product argument where ${\bf m}$ is the witness.
%----

The proof for an R1CS instance in Ligero constitutes of the check that verifies if the witness $\wit$ satisfies the following condition: $x = A \wit \land y = B \wit \land z = C \wit \land x \circ y = z$, where is $A, B, C$ matrices are dependent on the R1CS instance. The above check is segmented into three checks: (i) Interleaved check, (ii) Linear check, and (iii) Quadratic check. 
%---

In Ligero, $\prover$ rewrites the vectors $\wit, x, y, z$ as matrices in a canonical way, that is, the first $\ell$ entries are set as the first row of the matrix, next $\ell$ entries form the second row and so on. This way a matrix is formed which has, say $m$ rows and $\ell$ columns, where $m = \ell = O(\sqrt{N})$. This is referred to as the canonical matrix form. The prover encodes these matrices using Reed–Solomon(RS)-encoding to encode each row of the matrices.
In the compiled Ligero protocol, $\prover$ computes commitment of each column of these matrices. The commitment scheme takes $m$ length messages. In the original Ligero protocol, $t$ (depends on the query complexity) many columns and a linear combination of the rows are opened to the verifier. This enforces a bound on the proof size in Ligero. We circumvent this bound by replacing the openings with inner product checks. 
%---------------

In the interleaved check, $\prover$ proves that the rows of the oracle matrices are codewords. $\prover$ proves the claim probabilistically by showing that a random linear combination of the rows is a code word. $\verifier$ uniformly picks the coefficients for the linear combination.
Therefore, if one or more rows in the matrices are not correct codewords, then the linear combination is not a codeword with high probability. Furthermore, $\verifier$ queries the oracle with few indices (the number of queries is bounded by the security parameter $t$ for zero-knowledge) and receives the corresponding columns. Upon receiving the columns, $\verifier$ performs the same linear combination and checks that it is consistent with the row received from $\prover$.
%---
In the linear check, $\prover$ proves the knowledge of a vector $x$ such that $Ax = b$ holds for public matrix $A$ and vector $b$. This check is reduced to a probabilistic check $r^TAx = r^T b$, where $r$ is a random vector picked by $\verifier$. $\prover$ computes a polynomial $p(x)$ such that the sum of the evaluations on publicly known points is equal to $r^Tb$, and sends this $p(x)$ to $\verifier$. $\verifier$ checks if the above condition holds, and it further checks if $p(x)$ is constructed correctly from $r, A$, and $\wit$. $\verifier$ validates this by querying some locations of the oracle.
%---
In the quadratic check, $\prover$ proves that $x \circ y = z$ ($\circ$ denotes the Hadamard product of two vectors), where the corresponding encoded values are set as oracles. $\verifier$ gives a randomly sampled challenge vector $r$ to $\prover$. $\prover$ constructs a polynomial $p(x)$ using $r$ and the encoding polynomials of $x, y, z$. This polynomial should evaluate to $0$ on publicly known points. $\prover$ sends the polynomial $p(x)$ to $\verifier$. Upon receiving $p(x)$, $\verifier$ checks if the above condition is true or not. Furthermore, it checks whether $p(x)$ is correctly constructed. For this, it queries some locations of the oracles.
%---

In the compiled protocol, $\verifier$ obtains commitments corresponding to the queried columns, and instead of sending the openings of those columns, we perform the inner product argument. Since the opening of the columns is not sent, we can elongate the size of the columns without increasing the proof size. The inner product argument requires communication of $O(\log N)$ elements, where $N$ is the size of the witness vector. Therefore the newly obtained protocol has proof size $O(\ell + \log (m))$, where $\ell$ is the size of the row, and $m$ is the size of the column. However, verifying these inner products requires $O(N)$ exponentiations for a witness vector of dimension $N$. Thus increasing the column size unreasonably may result in a better proof size but increases the verification time, which is undesirable.
Furthermore, for the quadratic check protocol, the polynomial $p(x)$ depends non-linearly on the witness since it requires multiplication of the polynomials used for encoding $x, y$. Therefore, provers in $\Partyset$ engage in a secure computation protocol where they obtain an additive sharing of $p(x)$. In general, this incurs communication for computing a circuit of depth $1$ with $O(N)$ multiplications. Nevertheless, if the size of the shared circuit is sufficiently low (smaller than the row size in the canonical matrix), then it is possible to embed the shared circuit in a row. In this case, MPC is required only for the row that contains the shared circuit.

%------------------------------------------------------------
\paragraph*{$\name$ and $\dpname$:}
%----

In \name, we follow the same approach as in Ligero. We start with the extended witness $\wit$, consider it as a 3 dimensional matrix of size $p \times m \times s$ where $p$ 2 dimensional matrices of size $m \times s$, called a slice, are stacked one after another. In contrast to Ligero, where each row is encoded separately, we encode each two-dimensional slice separately. Let $\ell = s + \bi$, $h > 2m$, $n > 2\ell$ where $\bi$ is the bounded independence parameter depending on the query complexity. Let ${\bm \zeta} = \{ \zeta_1, \ldots, \zeta_{\ell}\}$, ${\bm \eta} = \{\eta_1, \ldots, \eta_{n}\}$ and ${\bm \alpha} = \{ \alpha_1, \ldots, \alpha_h \}$ be public set of points in $\FF$. We define $G = \{(\alpha_j,\zeta_k) : j \in [m],k \in [\ell] \}$ and $H = \{(\alpha_j,\eta_k) : j \in [h],k \in [n] \}$ to be the interpolation domain and evaluation domain respectively. To encode the $i$th slice, we construct a bivariate polynomial $Q^i(x,y)$ such that $Q^i(\alpha_j,\zeta_k) = \wit[i,j,k]$ and $\deg_x(Q^i) < m$ and $\deg_y(Q^i) < \ell$.
The encoded witness $\ewit$ is such that $\ewit[i,j,k] = Q^i(\alpha_j,\eta_k)$ for $i\in[p], j\in[h], k\in[n]$. Then each column of each slice is committed using Pedersen vector commitment. Here we instantiate the homomorphic $\FC$ with the Pedersen commitment scheme with message length $h$. Finally, $\verifier$ is provided oracle access to the committed matrix.
We design interleaved check protocol over committed values which is performed together with the linear check and quadratic check. The details are presented in Appendix~\ref{subsec:graphenec}. 
The linear check is similar to Ligero, where a probabilistic reduction is performed. $\verifier$ sends a random $r$ followed by both $\prover$ and $\verifier$ locally computing $R = r^TA$. Using ${\bm \alpha, \bm \eta}$ $\prover$, $\verifier$ interpolate $p$ polynomials $R^i(x,y)$ such that $R^i(\alpha_j, \zeta_k) = R[i,j,k]$ for $i\in[p], j\in[m], k\in[s]$ and $\deg_x(R^i) < m$, $\deg_y(R^i) < s$ for all $i$. $\prover$ computes polynomial $p_j(y) = \sum_{i\in[p]} R^i(\alpha_j, y) Q^i(\alpha_j, y)$. If the witness is correct then $r^TA\wit = r^Tb \Rightarrow \innp{R}{\wit} = r^T b \Rightarrow \sum_{j\in [m], k\in [s]} p_j (\zeta_k) = r^Tb$. Furthermore, $p_j(\eta_k) = \sum_{i\in[p]} R^i(\alpha_j, \eta) \ewit[i,j,k]$ due to the encoding. $\prover$ constructs a matrix $P$ of size $h \times n$ such that $P[j,k] = p_j(\eta)$ and commits to $P$. By providing partial opening using inner product arguments from Bulletproofs, $\prover$ proves that the polynomials $p_j(y)$ satisfies the following conditions:
\begin{itemize}
	%----
	\item $p_j(\eta_k) = \sum_{i\in[p]} R^i(\alpha_j, \eta) \ewit[i,j,k]$ for all $j \in [h], k \in [n]$
	%----
	\item $\sum_{j\in [m], k\in [s]} p_j (\zeta_k) = r^Tb$
	%----
\end{itemize}
%
Similar to the linear check, in the quadratic check, $\prover$ encodes $x,y,z$ to obtain $\ewit_x, \ewit_y, \ewit_z$ and lets $Q^i_x, Q^i_y, Q^i_z$  be the respective polynomials. $\prover$ constructs polynomials $Q^i$ such that $Q^i = Q^i_x\cdot Q^i_y - Q^i_z$. Since $x\circ y = z$, we have $Q^i(\alpha_j,\zeta_k) = 0$ for all $i\in[p], j\in[m], k\in[s]$. To check this, $\verifier$ sends a random vector $r\in \FF^p$ as a challenge. $\prover$ locally computes $p_j(\cdot) = \sum_{i \in [p]} r_i Q^i(\alpha_j, \cdot)$. Analogous to the linear check, $\prover$ forms $P$ matrix and commits to it, and further proves that $P$ satisfies the following:
\begin{itemize}
	%----
	\item $p_j(\eta_k) = \sum_{i\in[p]} r^i \left[\ewit_x[i,j,k] \cdot \ewit_y[i,j,k] - \ewit_z[i,j,k]\right]$ for all $j \in [h], k \in [n]$
	%----
	\item $p_j (\zeta_k) = 0$ for all $j \in [m], k \in [s]$
	%----
\end{itemize}
%----
%
To obtain $\dpname$, parties in $\Partyset$ execute secure multiplication to get an additive sharing of $\ewit_x[i,j,k] \cdot \ewit_y[i,j,k]$.The remaining steps in both linear and quadratic check do not require any interaction among the provers. Therefore, the provers interact to evaluate a circuit with $O(N)$ many multiplications with depth $1$. However, with smaller shared circuit, where the shared circuit can be embedded into a
column of a slice such that multiplication is required only for that column.
The details of the protocol are given in Appendix~\ref{app:grapehene_securityproofs}.


\subsection{Multiple oracle IOP}
\paragraph*{D-Aurora:}
In Aurora~\cite{aurora}, the size of the oracle is $O(N)$, the proof size is $O(\log^2 N)$, and number of rounds is $O(\log N)$. Aurora is an IOP-based proof system where almost all the messages from $\prover$ to $\verifier$ are set as oracles, and $\verifier$ makes oracle-queries to complete the verification.
%----
We can convert Aurora using our compiler, where we instantiate $\FC$ using Pedersen commitment, such that it supports DPZK. The compiled version retains the oracle size, proof size, and the number of rounds. However, the prover time and the verification time increase by $O(N\log N)$ group exponentiations due to oracle construction and validation costs. The distributed version needs secure evaluation of circuits with at least depth one and $O(N)$ multiplication gates in each round.
%----
\subsection{Non-IOP protocols}
\paragraph*{D-Bulletproofs:}
In Bulletproofs~\cite{bulletproofs}, the proof of an R1CS instance is reduced to an inner product argument which is proved recursively for the succinctness of the proof size. 
%----
The prover constructs two vector polynomials $l(X), r(X)$ of degree $1$ using the witness, statement and challenges from the verifier. It commits to the quadratic polynomial $t(X) = \innp{l(X)}{r(X)}$ where $t(X) = \sum_{i=0}^{d} \sum_{j=0}^{i} \innp{l_i}{r_j} X^{i+j}$ such that $l_i$ and $r_j$ are $l(X)$'s $i$th and $r(X)$'s $j$th vector coefficients respectively.
%----
The verifier sends a random point $x$ and prover obtains ${\bm l} = l(x)$, ${\bm r} = r(x)$ and $t = t(x)$. Using the inner product argument the prover proves that $\innp{{\bm l}}{{\bm r}} = t$.

In D-Bulletproofs, provers start with additive sharings of $l(X), r(X)$ and perform secure multiplications to obtain additive sharing of $t(X)$. This requires $O(N)$ multiplications. Upon receiving $x$, each prover locally obtains a sharing of ${\bm l,\bm r}, t$ and sends these values to the aggregator. The aggregator performs the inner product argument with the verifier to complete the proof.
%

\vspace*{-.3cm}
\paragraph*{D-Spartan:}
In Spartan~\cite{spartan}, an R1CS circuit is represented by 3 matrices $A, B, C$. A witness $\wit$ for the true instance satisfies $Az  \circ Bz = Cz$, where $z = (\stmt, 1, \wit)$ for public input $\stmt$. In Spartan, prover obtains low-degree polynomial extensions of $A, B, C, z$. Upon receiving the challenges from the verifier, the prover constructs a polynomial $f$. 
Then, a sum-check is performed to ensure that $f$ satisfies a consistency condition.

For the distributed proof generation, provers initiate the protocol with an additive sharing of the witness $\wit$ and achieve an additive sharing of $f$ by performing $O(N^2)$ secure multiplications. In Spartan, zero-knowledge is achieved by producing proofs of dot products. In the distributed setting, this either requires each prover to open its witness of the dot product to the aggregator or all the provers produce a distributed proof for the dot products. The latter case makes it a circular problem, whereas the former one has a privacy issue.
Even if a similar protocol could overcome the privacy issue, it would likely still suffer from the high communication complexity of $O(N^2)$ secure multiplication.
The above discussion further emphasizes the non-triviality of obtaining DPZK from any ZK protocol. 

%\begin{figure}[h!]
%	\centering
%	\resizebox{.3\textwidth}{!}{
%		\begin{tabular}{|c|c|c|} 
%			\hline
%			Protocols & $\prrounds$ & $\prcomm$ \\
%			\hline
%			D-Ligero & 1 			& $O(\sqrt{N})$\\
%			\hline
%			D-Aurora & $\log (N)$ 	& $O((N\log (N))$\\
%			\hline
%			D-Bulletproofs  & 1 		& $ O(N_s)$ \\
%			\hline
%			\dpname 		& 1 		& $ O( N^{1-2/c})$ \\
%			\hline
%		\end{tabular}
%	}
%	\caption{\footnotesize Comparison amongst the $\DPZK$s. Here $N$ is the size of the circuit and, $c$ is a positive integer of our choice. $\ast$ indicates non-private variant of Spartan.}\label{tab:DPZKwSmallN}
%\end{figure}
%
%Here we provide a comparison of $\DPZK$ protocols in the setting where the size of the shared circuit is low ($N_s < \sqrt{N}$) and the number of provers is small ($O(1)$). 