A zero-knowledge protocol tries to convince a verifier about the truth of a
statement without revealing any additional information. %Availability of
%practically efficient zero-knowledge protocols is enabling their real-life application. 
These include proving the correctness of
transactions in cryptocurrencies such as \cite{zerocash} or validating sensitive web browser data reported during
telemetry \cite{prio, MozillaPrio}. Still in the dream of a decentralized world, there are
multiple co-opetitive entities, i.e.,
collaborating but mutually distrusting entities interacting with each other to obtain insights
and maximize their goals. We envisage applications of zero-knowledge
proofs to enable these mutually
distrusting entities to prove a claim on their joint data. In this setting, the traditional zero-knowledge protocols are restrictive since they require
a single prover to have the entire witness needed to generate the proof. 

%\pdnote{do we want to motivate it in a way such that the honest prover setting we are looking at will suffice for the listed applications? }
In this work, we study the general setting of \textit{distributed prover
	zero-knowledge protocols} ($\DPZK$) where multiple co-opetitive entities, each possessing its own secret data, want to prove to a verifier that their secret data together satisfy a predicate of common interest. This is to be done without revealing any information
about their sensitive data to each other or to the verifier. 
%We illustrate through some real-world applications: %potential.

%------
%\begin{itemize}
%	%-----
%	\item[-] A simple but pertinent scenario is of a \textit{joint loan application} by an association of companies from a particular industry. The loan issuer has a set of financial requirements that it wants the association to satisfy. However, there is no single trusted entity to act as the prover to whom all the companies are willing to provide their sensitive business information.
%	%-----
%	\item[-] In cryptocurrency settings \cite{bitcoin, ethereum, zerocash}, this would enable a \textit{multi-wallet transaction} where the wallets are held by different parties or a \textit{proof of joint stake} where different parties hold the stake. These, in turn, enable secure collaboration applications by design on a blockchain network. % use it for joint auction-
%	%-----
%	\item[-] In trade logistics business networks \cite{scbn, e2open, tradelens}, a major reason for businesses to enter these networks is to benefit from cross-industry statistics. Publishing these statistics in a publicly verifiable manner without having a single trusted entity is another embodiment of this setting.
%	%-----
%	%	\item[-] In privacy-preserving server outsourced computation, it is assumed that the number of corrupted parties can be at most $t$, where $t<n/2$, for the honest majority setting and $t<n-1$ for the dishonest majority setting and $n$ is the number of servers. The protocols in these settings do not consider the scenario where the corruption exceeds $t$. In such a case, neither correctness nor privacy is ensured. So, suppose a user gets the output from the servers. In that case, there is no way to verify the correctness of the output, especially if the function is evaluated on inputs of multiple parties. In such a scenario, protocols can be enhanced by attaching a proof produced by all the servers stating the correctness of the computation. Here the statement would be the evaluated function, and the transcripts of each server become the share of the global witness. If the proof is valid, then the user is ensured that the output is indeed correct.
%	%-----
%	\item[-] \ADDED{More generally, in privacy-preserving server outsourced computations to multiple untrusted servers on sensitive data from data owner(s), it is typically assumed that the number of corrupt servers $s$ is $<n/2$ in the honest majority setting and $<n-1$ in the dishonest majority setting with $n$ being the number of servers. The protocols in these settings do not consider the scenario where the number of corrupt servers exceeds $s$, with neither correctness nor privacy ensured in that case. Hence, when a user gets the output from the servers, there is no way to verify the correctness of the output, especially if the function is evaluated on inputs of multiple data owners. In such a scenario, protocols can be enhanced by attaching a proof produced by all the servers, stating at least the correctness of the computation. Here the statement would be the evaluated function, and the transcripts of each server become the share of the global witness. If the proof is valid, then the user is ensured that the output is indeed correct. Observe that due to soundness of the proof, the proof for a wrong output will be rejected with overwhelming probability, even if all the servers are corrupt.}
%\end{itemize}

Formally, we have multiple provers $\prover_1, \ldots, \prover_{\Num}$
respectively possessing witnesses $\wit_1, \ldots, \wit_{\Num}$. For a predicate $C$ the provers wish to prove to a verifier that $C(\wit_1,\cdots,\wit_{\Num}) = 1$.  
A natural solution for distributed
proof generation is to start with the prover algorithm of a single prover
protocol and run this algorithm between multiple provers using multi-party computation (MPC). This generic construction was discussed by Pedersen~\cite{Ped92}. 
%{\color{blue}
However, the generic construction is unlikely to be efficient in practice. Known constructions of efficient zero-knowledge proofs involve
expensive computation (e.g. $\fft$) over complex mathematical structures such as fields, groups and elliptic curves, which are expensive when expressed
as an arithmetic circuit. Running multiparty computation over such circuits with several parties will be prohibitive. 
%\nnote{Any numbers to support the above?}
% An MPC implements each round of proof generation among the provers to
%generate the message to be sent to the verifier.
%The generic construction is optimal in proof size and verifier complexity in that it retains these complexities from the single-prover version irrespective of the number of provers. But, the complexity of proof generation suffers when the single prover protocol is not constructed with distributed proof generation in mind. Even an optimal $\round$-round proof generation algorithm with $O(N)$ multiplicative complexity of the prover's computation might result in a distributed proof generation algorithm with $\Omega(R \cdot N)$ communication among the provers.
To get around the inefficiency of a generic construction, some prior works have proposed efficient distributed proof generation for 
restricted class of computations.
These include simple predicates involved in threshold signatures \cite{DDS}, some sigma protocols \cite{EfficientTZ}, 
combined range proofs \cite{bulletproofs}. Distributed proof generation for more general computation has been considered in \cite{trinocchio}, but under
weaker trust model involving a trusted setup and majority of the parties
being honest \cite{trinocchio}. The goal of this paper is to enable DPZK for general computation without requiring trusted setup or honest majority. 
We summarize our contributions below.
%}
\paragraph*{Our contributions}
\begin{itemize}
	%----------
	\item We provide a formal definition of distributed prover zero-knowledge ($\DPZK$) in the real-ideal world paradigm. Furthermore, we identify and motivate relevant efficiency parameters to measure the efficiency of a $\DPZK$ protocol. %In the next section, we elaborate more on the new efficiency parameters.
	%----------
	\item \ADDED{We present a compiler that takes any IOP-based zero-knowledge protocol and converts it to an ``MPC-friendly'' zero-knowledge protocol 
	which can then be used to obtain a $\DPZK$ protocol.}
	%----------
	\item We illustrate the application of our compiler for two single-oracle IOPs, which is the preferred setting for our compiler.
	We obtain the protocol \textsf{D-Ligero} by using our compiler with the Ligero protocol from \cite{ligero}. Building upon the techniques
	used in Ligero, we construct a new single-oracle IOP which we call \name. The protocol \name\ admits smaller proof sizes
	than Ligero while ensuring efficient verification. Moreover we show that \name\ can also be efficiently compiled to yeild a DPZK protocol, 
	which we call \dpname. The zero knowledge protocols are described as interactive protocols secure against honest verifiers. Using standard
	transforms such as \cite{FS86, BCS16}, they can be used to obtain succint non-interactive arguments of knowledge (SNARGs), 
	which are secure against malicious verifiers in the Random Oracle model.
	%----------
\end{itemize}
Before providing a detailed overview of our work, we discuss potential real-world applications where distributed proof generation can be useful. 


\UPDATED{\section{Potential Applications}
\subsection{Multi-Wallet Anonymous Payments}\label{sec:application1}
One application scenario that we highlight is in the context of decentralized anonymous payment networks such as Zcash \cite{zerocash}. 
The users on the network create and consume coins via following transactions: (i) {\bf Mint} transaction which allows users to introduce coins to the system (after equivalent funds are
deposited to a backing pool), (ii) {\bf Spend} transaction which allows a user to consume his unspent coins and create new coins for other users and 
(iii) {\bf Redeem} transaction which allowed a user to redeem his unspent coins in exchange for equivalent funds in traditional banking systems. Our application 
is concerned with the {\bf Spend} transaction. In general an $n$-ary spend transaction consumes $n$ input coins and outputs $n$ coins with matching 
cumulative value. The output coins may be assigned to different users. To ensure unlinkability of input and output coins, coins are created
and spent using separate identifiers, known as {\em coin commitment} ($\mathsf{cm}$) and {\em serial number} ($\mathsf{sn}$) respectively. 
Roughly, $\mathsf{cm}$
and $\mathsf{sn}$ for a coin are linked via a trapdoor $\tau$: $\mathsf{cm} = f_1(\tau)$ and $\mathsf{sn}=f_2(\tau)$ for one-way functions
$f_1$ and $f_2$. Knowing one of them, it is infeasible to infer the other.  
The ledger maintains a list of commitments $\mathsf{CmList}$ for all the coins introduced to the system via {\bf Mint} or {\bf Spend} transactions. 
The ledger also maintains the list $\mathsf{SnList}$ of all the coins spent via {\bf Spend} or {\bf Redeem} transactions. As part of a spend transaction, 
the user supplies serial numbers $\mathsf{sn}_1,\ldots,\mathsf{sn}_n$ for the $n$ coins being consumed along with the commitments for the output coins.
 Additionally, the 
user supplies a zero knowledge proof $\pi$ attesting knowledge of $\{\mathsf{cm}_i,\tau_i\}_{i=1}^n$ such that:
\begin{itemize}
    \item $\mathsf{cm}_i\in \mathsf{CmList}$ for all $i\in [n]$.
    \item $\mathsf{cm}_i=f_1(\tau_i)$ and $\mathsf{sn}_i=f_2(\tau_i)$ for all $i\in [n]$.
\end{itemize}
By verifying the above zero knowledge proof, the participants on the network ensure that serial numbers correspond to coins that were legitimately 
created and by further verifying that $\mathsf{sn}_i\not\in \mathsf{SnList}$ for all $i\in [n]$, they ensure that the same coin is not 
spent more than once (double spending).

 The above setup does not allow different users to pool in their coins as inputs to a spend transaction, for example,
to jointly pay off another entity. The limitation stems from having to generate a zero knowledge proof, for which the user requires knowledge of all the
private information associated with a coin. One way to work around the limitation is for users to transfer their respective coins (via spend transactions) 
to one designated user, who then initiates a spend transaction consuming all the received coins on their behalf. 
However, the operation is no longer {\em atomic} and the designated user may not go ahead with his transaction. 

Using distributed proof generation, the set of pooling users supply the share of the witness corresponding to the coins they own,
and use a distributed proof generation protocol proposed in this paper to output a proof that proves the validity of all the coins to the 
larger network. This results in an atomic payment between upto $n$ senders and recipients.

\subsection{Cryptocurrency Based Auctions}\label{sec:application2}
Closely related to previous application is an auction based on cryptocurrencies, where we again use Zcash as an example. 
We provide a simplified description where an auction for an item invites bids from participants. After an announced time interval, the bidder with maximum bid wins the auction. The bids are
made using cryptocurrency, where the bidder is required to prove that it owns currency amounting to its bid. The key advantage of a decentralized
cryptocurrency is that bidders can prove their ``stake'' without revealing additional details. More precisely, an auctioneer announces an auction
$\adv$ with a time-out block height as $T$ (i.e, the bidding ends at block height $T$). A bid is made using a special transaction of type {\bf Bid}
consisting of the vector $\bm{s}=(\mathsf{sn}_1,\ldots,\mathsf{sn}_n)$ containing serial numbers of unspent coins, integer $v$ denoting the cumulative
value of the coins (bidding amount) and a zero knowledge proof $\pi$ attesting that the coins are legitimate and add up to $v$. The transaction succeeds
if the proof $\pi$ verifies and serial numbers in $\bm{s}$ do not appear in the list $\mathsf{SnList}$. A successful bid transaction also marks
the serial numbers in $\bm{s}$ in a {\tt LOCKED} state (i.e, they are not spendable till block-height $T$). Post the time-out $T$, 
the bid with the maximum value $v$ is selected as the winning bid. The 
coins $\bm{s}$ in it are added to the list $\mathsf{SnList}$, while {\tt LOCKED} coins from other bids are {\tt UNLOCKED} (i.e, they are spendable again).
In this description, we leave out the details of further engagement between the auctioneer and the winning participants, as that is external to the network.
Leveraging distributed proof generation allows a set of participants to pool in their coins and place a joint bid in the auction. The distributed proof
generation in this case is similar to that in the previous application.


}
%\pnote{old version below}
%We design zero-knowledge protocols which enable efficient distributed proof generation, informally an ``MPC-friendly'' proof generation, while supporting \textit{arbitrary} predicates. %The proof generation algorithm is usually measured for efficiency in terms of the number of group operations and the number of rounds of interaction with the verifier.
%Our work first identifies and motivates relevant efficiency parameters that a
%zero knowledge protocol should meet to admit efficient distributed proof generation. 
%We then construct a zero-knowledge 
%protocol which is the state-of-art in these parameters. The discussions will
%focus on {\em public-coin honest-verifier} protocols with a transparent setup,
%which is essentially the setting for real-world applications we envisage: 1) protocols with
%such verifiers can be converted to succinct non-interactive  arguments
%(zk-SNARGs)  via the Fiat-Shamir and like transforms \cite{FS86, BCS16}, and  2)
%generating the parameters of the protocol should ideally {\em not} involve a
%trusted third party. The next few subsections elaborate on the new efficiency
%parameters, formal definition and our new construction for $\DPZK$ that leverages 
%a new zero-knowledge argument constructed in this paper, which may be of
%independent interest.